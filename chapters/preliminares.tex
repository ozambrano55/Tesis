% Portada
\begin{titlepage}
    \begin{center}
        \vspace*{0.5cm}
        
        \Huge
        \textbf{UNIVERSIDAD TECNOLÓGICA ECOTEC}
        
        \vspace{1cm}
        
        \Large
        MAESTRÍA EN SISTEMAS DE INFORMACIÓN CON MENCIÓN EN INTELIGENCIA DE NEGOCIOS
        
        \vspace{0.5 cm}
        
        \includegraphics[width=0.4\textwidth]{images/logo.jpg}
        
        \vspace{0.5 cm}
        
        \Huge
        \textbf{Desarrollo de Modelo predictivo basado en machine learning para la identificación temprana de riesgo de morosidad en carteras crediticias}
        
        \vspace{0.5 cm}
        
        \Large
        \textbf{PROYECTO DE DESARROLLO}
        
        \vspace{0.5 cm}
        
        \Large
        \textbf{Autor:}\\
        Orlando Daniel Zambrano Zúñiga\\
        
        \vspace{0.5cm}
        
        \textbf{Tutor:}\\
        Magíster Christian Merchán\\
        
        \vfill
        
        \Large
        Guayaquil, Ecuador\\
        2025
    \end{center}
\end{titlepage}

% Declaración de autoría original
\chapter*{Declaración de Autoría Original}
\thispagestyle{empty}

Por la presente declaro que soy el autor de este Trabajo de Titulación titulado ``Desarrollo de Modelo predictivo basado en machine learning para la identificación temprana de riesgo de morosidad en carteras crediticias'' que se ha presentado a la Universidad Tecnológica ECOTEC para obtener el título de Máster en Sistemas de Información con Mención en Inteligencia de Negocios.

Este trabajo es original y se ha desarrollado respetando los derechos intelectuales de terceros, conforme las citas que constan en el documento y cuyas fuentes se incorporan en las referencias bibliográficas.

En virtud de esta declaración, me responsabilizo del contenido, veracidad y alcance científico del trabajo de titulación referido.

\vspace{2cm}

\begin{flushright}
\_\_\_\_\_\_\_\_\_\_\_\_\_\_\_\_\_\_\_\_\_\_\_\_\_\_\_\\
Orlando Daniel Zambrano Zúñiga\\
C.I.: 0920470762\\
Fecha: \today
\end{flushright}

\newpage

% Dedicatoria
\chapter*{Dedicatoria}
\thispagestyle{empty}

\begin{flushright}
\emph{A mi familia, por su apoyo incondicional\\
durante todo este proceso académico.\\
Su paciencia y aliento han sido fundamentales\\
para la culminación de este proyecto.}
\end{flushright}

\newpage

% Agradecimientos
\chapter*{Agradecimientos}
\thispagestyle{empty}

Quiero expresar mi más sincero agradecimiento a todas las personas que hicieron posible la realización de este trabajo:

A mi tutor, Magíster Christian Merchán, por su guía, asesoramiento y crítica constructiva a lo largo de todo el proceso de investigación.

A la Universidad Tecnológica ECOTEC y a todos los docentes de la maestría, por compartir sus conocimientos y experiencias, contribuyendo significativamente a mi formación profesional.

A la empresa que me permitió acceder a sus datos, por la confianza depositada y la apertura para implementar el proyecto desarrollado.

A mis compañeros de maestría, por el enriquecedor intercambio de ideas y el apoyo mutuo durante estos años de estudio.

A todos quienes, de una u otra forma, contribuyeron al desarrollo exitoso de este trabajo.

\newpage

% Resumen
\chapter*{Resumen}
\thispagestyle{empty}

El presente trabajo aborda el desarrollo e implementación de un modelo predictivo basado en técnicas de machine learning para la identificación temprana del riesgo de morosidad en carteras crediticias del sector mayorista. Utilizando datos históricos del período 2017-2024 de una empresa distribuidora de productos de juguetería, hogar, aseo y cocina, se aplicó la metodología CRISP-DM para estructurar el proceso de minería de datos y desarrollo del modelo.

El estudio integra variables tradicionales de comportamiento de pago con factores contextuales como ubicación geográfica y estacionalidad, frecuentemente omitidos en modelos genéricos. Mediante la evaluación comparativa de múltiples algoritmos, se identificó XGBoost como la técnica con mejor rendimiento, alcanzando una precisión del 84\% en la predicción de riesgo de morosidad, superando significativamente los métodos tradicionales previamente utilizados.

La implementación del sistema incluyó el desarrollo de un dashboard interactivo y un sistema de alertas tempranas que amplió el tiempo promedio de anticipación de 3 a 15 días, permitiendo intervenciones preventivas efectivas que resultaron en una reducción del 18\% en la tasa general de morosidad durante el periodo piloto.

El análisis de costo-beneficio demuestra un retorno de inversión estimado de 6 meses, validando la viabilidad económica de la implementación de soluciones basadas en inteligencia de negocios en el contexto empresarial ecuatoriano.

\textbf{Palabras clave:} Machine learning, riesgo crediticio, morosidad, análisis predictivo, sector mayorista, cartera crediticia, CRISP-DM.

\newpage

% Abstract (en inglés)
\chapter*{Abstract}
\thispagestyle{empty}

This work addresses the development and implementation of a predictive model based on machine learning techniques for the early identification of default risk in credit portfolios in the wholesale sector. Using historical data from the 2017-2024 period from a distribution company of toys, household, cleaning, and kitchen products, the CRISP-DM methodology was applied to structure the data mining process and model development.

The study integrates traditional payment behavior variables with contextual factors such as geographic location and seasonality, frequently omitted in generic models. Through comparative evaluation of multiple algorithms, XGBoost was identified as the technique with the best performance, achieving 84\% accuracy in predicting default risk, significantly outperforming previously used traditional methods.

The system implementation included the development of an interactive dashboard and an early warning system that extended the average anticipation time from 3 to 15 days, allowing effective preventive interventions that resulted in an 18\% reduction in the overall delinquency rate during the pilot period.

The cost-benefit analysis demonstrates an estimated return on investment of 6 months, validating the economic viability of implementing business intelligence solutions in the Ecuadorian business context.

\textbf{Keywords:} Machine learning, credit risk, default, predictive analysis, wholesale sector, credit portfolio, CRISP-DM.