\chapter{Conclusiones y Recomendaciones}

\section{Conclusiones}
A partir de los resultados obtenidos en el desarrollo e implementación del modelo predictivo para la identificación temprana del riesgo de morosidad, se derivan las siguientes conclusiones:

\begin{enumerate}
    \item El desarrollo del modelo predictivo basado en técnicas de machine learning ha permitido alcanzar una precisión del 84\% en la identificación temprana del riesgo de morosidad, superando significativamente el objetivo establecido del 80\% y mejorando en un 23\% la precisión del método tradicional anteriormente utilizado.
    
    \item La integración de variables geográficas y estacionales en el modelo predictivo ha demostrado ser fundamental, representando más del 20\% del poder predictivo total. Esto confirma la hipótesis inicial sobre la importancia de estos factores en el contexto del comercio mayorista ecuatoriano.
    
    \item La metodología CRISP-DM ha demostrado ser un marco efectivo para estructurar el proceso de desarrollo del modelo predictivo, permitiendo un enfoque sistemático desde la comprensión del negocio hasta la implementación y evaluación de resultados.
    
    \item La segmentación previa de la cartera mediante técnicas de clustering (K-Means) ha facilitado la identificación de cuatro perfiles distintivos de clientes (Premium, Estable, Estacional y Alto riesgo), lo que permite una aplicación más precisa y contextualizada de los modelos predictivos.
    
    \item El algoritmo XGBoost ha mostrado el mejor rendimiento entre las técnicas evaluadas, con métricas consistentemente superiores en precisión, sensibilidad, especificidad y área bajo la curva ROC.
    
    \item La implementación del sistema de alertas tempranas ha permitido ampliar el tiempo promedio de anticipación de 3 a 15 días, lo que ha facilitado intervenciones preventivas efectivas que resultaron en una reducción del 18\% en la tasa general de morosidad durante el periodo piloto.
    
    \item El análisis de importancia de características ha revelado que la variabilidad en el comportamiento de pago (representada por variables como Variabilidad\_Pago y Máximo\_Días\_Atraso) posee mayor poder predictivo que indicadores estáticos tradicionales, lo que sugiere la necesidad de un enfoque dinámico en la evaluación crediticia.
    
    \item El dashboard interactivo implementado ha facilitado la interpretación y utilización de las predicciones generadas por el modelo, permitiendo que usuarios con diferentes niveles de experticia técnica puedan beneficiarse de los resultados predictivos.
    
    \item El análisis costo-beneficio revela un retorno de inversión estimado de 6 meses, considerando tanto los beneficios directos (reducción de incobrables) como indirectos (optimización de procesos operativos).
    
    \item La reducción del 64\% en falsos positivos respecto al método tradicional representa una mejora significativa en la eficiencia operativa, al disminuir el tiempo y recursos dedicados a casos incorrectamente identificados como de alto riesgo.
\end{enumerate}

\section{Recomendaciones}

Con base en la experiencia adquirida durante el desarrollo e implementación del proyecto, así como en las limitaciones identificadas, se plantean las siguientes recomendaciones:

\begin{enumerate}
    \item \textbf{Expansión del conjunto de datos:} Incorporar progresivamente datos de nuevos clientes y transacciones para fortalecer la capacidad predictiva del modelo, especialmente para el segmento "Estacional" donde se identificaron mayores desafíos.
    
    \item \textbf{Integración de variables externas:} Complementar el modelo con indicadores macroeconómicos regionales y sectoriales que podrían capturar factores externos no considerados actualmente.
    
    \item \textbf{Refinamiento por segmento:} Desarrollar modelos específicos para cada uno de los segmentos identificados, optimizando los parámetros y variables según las características particulares de cada grupo.
    
    \item \textbf{Implementación de aprendizaje continuo:} Establecer un proceso automatizado de reentrenamiento periódico que incorpore nuevos datos y resultados de intervenciones, permitiendo que el modelo evolucione y se adapte a cambios en patrones crediticios.
    
    \item \textbf{Ampliación del periodo de validación:} Extender el periodo de validación a un ciclo anual completo para evaluar con mayor precisión el desempeño del modelo ante la estacionalidad característica del sector.
    
    \item \textbf{Desarrollo de interfaces de integración:} Implementar APIs adicionales que faciliten la integración del sistema predictivo con otras plataformas empresariales, como sistemas de gestión de relaciones con clientes (CRM) y planificación de recursos empresariales (ERP).
    
    \item \textbf{Capacitación avanzada a usuarios:} Desarrollar un programa de formación continua para los usuarios del sistema, abordando tanto aspectos técnicos de interpretación de resultados como estrategias de intervención basadas en las predicciones.
    
    \item \textbf{Exploración de técnicas explicativas:} Implementar métodos como SHAP (SHapley Additive exPlanations) o LIME (Local Interpretable Model-agnostic Explanations) para mejorar la interpretabilidad de las predicciones individuales, facilitando la explicación de resultados a clientes y stakeholders.
    
    \item \textbf{Estudios comparativos sectoriales:} Promover colaboraciones con otras empresas del sector para realizar benchmarking de indicadores de riesgo crediticio y validar la generalización del modelo en contextos similares.
    
    \item \textbf{Desarrollo de módulos predictivos complementarios:} Expandir el alcance del sistema con módulos adicionales enfocados en predecir el monto probable de recuperación y el tiempo estimado hasta la normalización en casos de morosidad.
\end{enumerate}

\section{Trabajo Futuro}

El presente proyecto establece las bases para diversas líneas de investigación y desarrollo futuro:

\begin{enumerate}
    \item \textbf{Modelos híbridos:} Explorar la combinación de diferentes técnicas de machine learning con enfoques de análisis de redes sociales y análisis de texto para incorporar datos no estructurados disponibles en interacciones con clientes.
    
    \item \textbf{Sistemas de recomendación:} Desarrollar algoritmos que no solo predigan el riesgo de morosidad, sino que recomienden estrategias específicas de intervención basadas en características del cliente y patrones históricos de efectividad.
    
    \item \textbf{Análisis predictivo multidimensional:} Expandir el enfoque predictivo para abordar simultáneamente múltiples aspectos del comportamiento del cliente, incluyendo patrones de compra, sensibilidad a promociones y potencial de crecimiento, creando un perfil integral que complemente la evaluación de riesgo.
    
    \item \textbf{Aplicación de aprendizaje por refuerzo:} Implementar técnicas de aprendizaje por refuerzo para optimizar dinámicamente las estrategias de intervención, permitiendo que el sistema aprenda de la efectividad de acciones previas.
    
    \item \textbf{Extensión a otros sectores:} Adaptar la metodología y arquitectura desarrolladas para su aplicación en otros sectores con dinámicas crediticias similares, como distribución mayorista de alimentos, material de construcción o equipamiento tecnológico.
    
    \item \textbf{Incorporación de tecnologías emergentes:} Explorar la integración de tecnologías como blockchain para mejorar la transparencia y trazabilidad de las transacciones crediticias, o el Internet de las Cosas (IoT) para capturar datos operativos relevantes de clientes mayoristas.
\end{enumerate}

\section{Contribuciones Principales}

Las contribuciones más significativas de este trabajo pueden resumirse en:

\begin{enumerate}
    \item El desarrollo de un modelo predictivo con alta precisión (84\%) adaptado específicamente a las particularidades del sector mayorista ecuatoriano, considerando factores regionales y estacionales frecuentemente omitidos en modelos genéricos.
    
    \item La validación empírica de la efectividad de técnicas avanzadas de machine learning como XGBoost en un contexto con restricciones de datos (cartera limitada de clientes) y alta heterogeneidad.
    
    \item La integración exitosa del modelo predictivo en la operación cotidiana de la empresa mediante un sistema de información completo que abarca desde la captura y procesamiento de datos hasta la visualización interactiva y generación de alertas.
    
    \item La cuantificación del impacto económico y operativo derivado de la implementación de técnicas de inteligencia de negocios, demostrando un retorno de inversión tangible y contribuyendo a la justificación de inversiones similares en el sector.
    
    \item La documentación metodológica detallada del proceso de desarrollo, que puede servir como referencia para proyectos similares en el contexto de empresas mayoristas ecuatorianas y latinoamericanas.
\end{enumerate}

Este trabajo demuestra que la aplicación de técnicas avanzadas de machine learning y análisis predictivo puede generar beneficios significativos incluso en contextos empresariales con recursos tecnológicos limitados, siempre que se adapten adecuadamente a las particularidades del sector y se integren efectivamente en los procesos operativos existentes.