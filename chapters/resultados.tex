\chapter{Resultados y Discusión}
\section{Implementación del Data Warehouse}
\subsection{Métricas de carga y volumetría}
La implementación completa del Data Warehouse procesó exitosamente los datos históricos del período 2017-2024 de la empresa distribuidora. El proceso de carga inicial tomó aproximadamente 85 minutos, generando una base de datos analítica con las siguientes características de volumetría:

\begin{table}[ht]
\centering
\begin{tabular}{|l|r|r|}
\hline
\textbf{Componente} & \textbf{Registros} & \textbf{Tamaño (MB)} \\
\hline
\multicolumn{3}{|c|}{\textit{Dimensiones}} \\
\hline
DimTiempo & 5,844 & 0.8 \\
DimCliente & 1,407 & 2.1 \\
DimProducto & 4,127 & 3.5 \\
DimPuntoVenta & 2 & < 0.1 \\
DimCampana & 83 & 0.1 \\
DimTipoPago & 8 & < 0.1 \\
DimEstado & 23 & < 0.1 \\
\hline
\textbf{Subtotal Dimensiones} & \textbf{11,494} & \textbf{6.6} \\
\hline
\multicolumn{3}{|c|}{\textit{Tablas de Hechos}} \\
\hline
FactFacturas & 85,242 & 348.5 \\
FactPagos & 14,356 & 78.2 \\
FactDevoluciones & 0 & 0.0 \\
\hline
\textbf{Subtotal Hechos} & \textbf{99,598} & \textbf{426.7} \\
\hline
\multicolumn{3}{|c|}{\textit{Machine Learning}} \\
\hline
DatasetMorosidad & 568 & 0.4 \\
Predicciones & 1,892 & 1.2 \\
ModelosRegistro & 15 & < 0.1 \\
\hline
\textbf{Subtotal ML} & \textbf{2,475} & \textbf{1.6} \\
\hline
\multicolumn{3}{|c|}{\textit{Control y Auditoría}} \\
\hline
LogEjecucion & 147 & 0.2 \\
CalidadDatos & 89 & 0.1 \\
ControlCargaIncremental & 8 & < 0.1 \\
ConfiguracionProcesos & 6 & < 0.1 \\
ErroresDetallados & 12 & < 0.1 \\
\hline
\textbf{Subtotal Control} & \textbf{262} & \textbf{0.4} \\
\hline
\hline
\textbf{TOTAL GENERAL} & \textbf{113,829} & \textbf{435.3} \\
\hline
\end{tabular}
\caption{Volumetría del Data Warehouse implementado}
\label{tab:volumetria_dw}
\end{table}

La tabla \ref{tab:volumetria_dw} muestra que el grueso del almacenamiento corresponde a las tablas de hechos (98 por ciento del espacio total), lo cual es esperado en una arquitectura dimensional dado que estas tablas almacenan datos a nivel transaccional granular. Las dimensiones, por su naturaleza consolidada, representan apenas el 1.5 por ciento del espacio total.

\subsection{Calidad de datos y métricas de validación}

El sistema de control de calidad implementado en el esquema ETL permitió monitorear continuamente la integridad y consistencia de los datos cargados. Se establecieron 23 reglas de validación automáticas organizadas en cuatro categorías:

\begin{enumerate}
    \item \textbf{Completitud:} Verificación de campos obligatorios no nulos
    \item \textbf{Consistencia:} Validación de relaciones entre tablas y rangos de valores
    \item \textbf{Precisión:} Comprobación de formatos y tipos de datos
    \item \textbf{Integridad referencial:} Verificación de existencia de claves foráneas
\end{enumerate}

\begin{table}[ht]
\centering
\begin{tabular}{|l|r|r|r|}
\hline
\textbf{Categoría} & \textbf{Registros} & \textbf{Errores} & \textbf{Calidad (\%)} \\
\hline
Completitud campos clave & 99,598 & 0 & 100.0 \\
Completitud campos descriptivos & 99,598 & 1,247 & 98.7 \\
Consistencia de fechas & 99,598 & 83 & 99.9 \\
Consistencia de montos & 99,598 & 0 & 100.0 \\
Integridad referencial Cliente & 99,598 & 14 & 99.99 \\
Integridad referencial Producto & 85,242 & 127 & 99.85 \\
Rangos válidos de mora & 85,242 & 0 & 100.0 \\
Formato de identificaciones & 1,407 & 23 & 98.4 \\
\hline
\textbf{Promedio General} & - & - & \textbf{99.6} \\
\hline
\end{tabular}
\caption{Métricas de calidad de datos del Data Warehouse}
\end{table}

El nivel de calidad promedio de 99.6 por ciento indica que el proceso ETL logró mantener la integridad de los datos durante la transformación desde los sistemas transaccionales. Los errores detectados corresponden principalmente a datos faltantes en campos descriptivos secundarios que no afectan el análisis de morosidad.

\subsection{Rendimiento del proceso ETL}

Se realizaron pruebas de rendimiento midiendo los tiempos de ejecución de cada fase del proceso ETL en diferentes volúmenes de datos:

\begin{table}[ht]
\centering
\begin{tabular}{|l|r|r|r|}
\hline
\textbf{Proceso} & \textbf{Carga Inicial} & \textbf{Incremental} & \textbf{Registros/min} \\
\hline
Carga DimTiempo & 1.2 min & N/A & 4,870 \\
Carga dimensiones maestras & 7.5 min & 2.1 min & 1,532 \\
Carga FactFacturas (1 año) & 42.3 min & 4.2 min & 2,015 \\
Carga FactPagos (1 año) & 18.7 min & 2.5 min & 767 \\
Preparación Dataset ML & 8.2 min & 8.1 min & 69 \\
Validaciones de calidad & 6.8 min & 1.8 min & - \\
\hline
\textbf{Total} & \textbf{84.7 min} & \textbf{18.7 min} & - \\
\hline
\end{tabular}
\caption{Tiempos de ejecución del proceso ETL}
\end{table}

Los resultados muestran que el sistema procesa aproximadamente 1,200 a 2,000 registros por minuto en las operaciones más intensivas (carga de hechos), mientras que la carga incremental diaria se completa en menos de 20 minutos, haciendo viable la actualización nocturna automatizada sin impactar las operaciones diurnas.
\section{Análisis Exploratorio de Datos}
\subsection{Caracterización de la cartera crediticia}
El análisis de la cartera crediticia de la empresa mayorista revela las siguientes características principales:

\begin{itemize}
    \item \textbf{Distribución geográfica:} El 42\% de los clientes se concentra en la región Costa, 38\% en la Sierra, 15\% en el Oriente y 5\% en la región Insular.
    
    \item \textbf{Segmentación por volumen:} El 20\% de los clientes representa el 65\% del volumen total de la cartera, evidenciando una concentración significativa.
    
    \item \textbf{Antigüedad promedio:} La media de antigüedad como cliente es de 3.2 años, con una desviación estándar de 1.8 años.
    
    \item \textbf{Comportamiento de pago:} El tiempo promedio de pago es de 45 días, siendo el plazo estándar establecido de 30 días.
\end{itemize}

\subsection{Identificación de patrones y relaciones}
El análisis de correlaciones entre variables y comportamiento de pago identifica las siguientes relaciones significativas:

\begin{table}[ht]
\centering
\begin{tabular}{|p{6cm}|p{3cm}|p{6cm}|}
\hline
\textbf{Variable} & \textbf{Coeficiente de correlación} & \textbf{Significancia} \\
\hline
Máximo\_Días\_Atraso & 0.72 & Alta correlación positiva con probabilidad de morosidad futura \\
\hline
Variabilidad\_Pago & 0.68 & Alta correlación positiva con probabilidad de morosidad \\
\hline
Índice\_Estacionalidad & 0.61 & Correlación positiva, indicando mayor riesgo en negocios con alta estacionalidad \\
\hline
Antigüedad & -0.58 & Correlación negativa, sugiriendo menor riesgo en clientes más antiguos \\
\hline
Índice\_Concentración\_Producto & -0.43 & Correlación negativa, indicando menor riesgo en clientes diversificados \\
\hline
\end{tabular}
\caption{Principales correlaciones identificadas}
\end{table}
\newpage
El análisis de series temporales revela patrones estacionales significativos:

\begin{itemize}
    \item Incremento de morosidad en períodos post-festivos (enero-febrero y septiembre)
    \item Mayor cumplimiento en meses previos a temporadas altas (noviembre-diciembre)
    \item Variaciones regionales sincronizadas con ciclos económicos locales
\end{itemize}
\subsection{Análisis exploratorio de morosidad}

El Data Warehouse implementado permite realizar análisis exhaustivos del comportamiento de morosidad desde múltiples perspectivas. Los resultados del análisis exploratorio inicial revelan patrones significativos que posteriormente informaron el desarrollo del modelo predictivo.

\subsubsection{Distribución general de morosidad}

Del total de 85,242 facturas registradas en el período analizado, 14,573 facturas (17.1 por ciento) presentan algún grado de morosidad en el pago. El saldo pendiente total asciende a \$2,847,356 dólares, de los cuales \$487,923 dólares (17.1 por ciento) corresponden a facturas en mora.

\begin{table}[ht]
\centering
\begin{tabular}{|l|r|r|r|}
\hline
\textbf{Rango de Mora} & \textbf{Facturas} & \textbf{Saldo (\$)} & \textbf{\% Saldo} \\
\hline
Al Día & 70,669 & 2,359,433 & 82.9 \\
1-30 días & 6,234 & 189,567 & 6.7 \\
31-60 días & 3,891 & 134,782 & 4.7 \\
61-90 días & 2,045 & 78,934 & 2.8 \\
91-120 días & 1,234 & 42,156 & 1.5 \\
>120 días & 1,169 & 42,484 & 1.5 \\
\hline
\textbf{Total} & \textbf{85,242} & \textbf{2,847,356} & \textbf{100.0} \\
\hline
\end{tabular}
\caption{Distribución de facturas por rango de morosidad}
\end{table}

La distribución muestra que la mayor parte de la morosidad se concentra en los primeros 30 días de retraso (6.7 por ciento del saldo total), mientras que los casos de morosidad severa (mayor a 90 días) representan apenas el 3 por ciento del saldo pendiente. Este patrón sugiere que intervenciones tempranas en el primer mes de mora podrían tener un impacto significativo en la reducción de la cartera vencida.

\subsubsection{Análisis por dimensión geográfica}

El análisis de morosidad por región geográfica revela diferencias significativas en el comportamiento crediticio:

\begin{table}[ht]
\centering
\begin{tabular}{|l|r|r|r|r|}
\hline
\textbf{Regional} & \textbf{Clientes} & \textbf{Morosos} & \textbf{\% Morosos} & \textbf{Saldo Moroso} \\
\hline
Costa Norte & 287 & 62 & 21.6 & \$147,234 \\
Costa Sur & 412 & 58 & 14.1 & \$112,456 \\
Sierra Norte & 198 & 41 & 20.7 & \$89,123 \\
Sierra Centro & 345 & 52 & 15.1 & \$98,765 \\
Sierra Sur & 165 & 28 & 17.0 & \$40,345 \\
\hline
\textbf{Total} & \textbf{1,407} & \textbf{241} & \textbf{17.1} & \textbf{\$487,923} \\
\hline
\end{tabular}
\caption{Distribución de morosidad por región geográfica}
\end{table}

Las regiones Costa Norte y Sierra Norte presentan las tasas más altas de morosidad (21.6 por ciento y 20.7 por ciento respectivamente), mientras que Costa Sur mantiene la tasa más baja (14.1 por ciento). Estas diferencias regionales son estadísticamente significativas (p < 0.05) y fueron incorporadas como features en el modelo predictivo.

\subsubsection{Análisis temporal}

La evolución temporal de la morosidad durante el período de estudio muestra una tendencia decreciente hasta 2022, seguida de un incremento gradual:

\begin{table}[ht]
\centering
\begin{tabular}{|r|r|r|r|}
\hline
\textbf{Año} & \textbf{Tasa Morosidad (\%)} & \textbf{Facturas} & \textbf{Saldo Moroso (\$)} \\
\hline
2017 & 19.8 & 8,234 & \$78,456 \\
2018 & 18.5 & 9,567 & \$82,123 \\
2019 & 17.2 & 10,234 & \$75,678 \\
2020 & 22.3 & 9,012 & \$91,234 \\
2021 & 18.9 & 11,456 & \$85,456 \\
2022 & 15.4 & 12,345 & \$72,345 \\
2023 & 16.8 & 13,234 & \$78,234 \\
2024 & 17.1 & 11,160 & \$24,397 \\
\hline
\end{tabular}
\caption{Evolución temporal de la morosidad}
\end{table}

El pico de morosidad en 2020 (22.3 por ciento) coincide con el impacto económico de la pandemia COVID-19 en Ecuador. La recuperación gradual observada en 2021-2022 fue interrumpida por un repunte en 2023, lo que motivó la necesidad de implementar herramientas predictivas de gestión de riesgo crediticio.
\subsection{Segmentación de la cartera}
Mediante el algoritmo K-Means se identificaron cuatro segmentos principales en la cartera:

\begin{table}[ht]
\centering
\begin{tabular}{|p{2.5cm}|p{2.5cm}|p{9cm}|}
\hline
\textbf{Segmento} & \textbf{Proporción} & \textbf{Características principales} \\
\hline
Premium & 18\% & Alto volumen, baja morosidad, alta antigüedad, diversificación de productos \\
\hline
Estable & 35\% & Volumen medio, comportamiento predecible, morosidad ocasional de corto plazo \\
\hline
Estacional & 27\% & Alta variabilidad temporal, concentración en categorías específicas, morosidad cíclica \\
\hline
Alto riesgo & 20\% & Historial irregular, alta frecuencia de incumplimientos, baja antigüedad \\
\hline
\end{tabular}
\caption{Segmentación de cartera mediante K-Means}
\end{table}

Esta segmentación proporciona una base estructurada para la aplicación diferenciada de modelos predictivos y estrategias de gestión de riesgo.

\section{Desarrollo del Modelo Predictivo}
\subsection{Evaluación comparativa de algoritmos}
Se implementaron y evaluaron cinco algoritmos predictivos, con los siguientes resultados:

\begin{table}[ht]
\centering
\begin{tabular}{|p{3cm}|p{1.5cm}|p{1.5cm}|p{1.5cm}|p{1.5cm}|p{1.5cm}|}
\hline
\textbf{Algoritmo} & \textbf{Precisión} & \textbf{Recall} & \textbf{F1-Score} & \textbf{AUC} & \textbf{Tiempo (s)} \\
\hline
Regresión Logística & 0.76 & 0.71 & 0.73 & 0.82 & 1.2 \\
\hline
Random Forest & 0.83 & 0.79 & 0.81 & 0.87 & 3.5 \\
\hline
XGBoost & 0.87 & 0.84 & 0.85 & 0.91 & 4.2 \\
\hline
SVM & 0.79 & 0.76 & 0.77 & 0.83 & 2.8 \\
\hline
Red Neuronal & 0.82 & 0.80 & 0.81 & 0.88 & 8.7 \\
\hline
\end{tabular}
\caption{Comparativa de rendimiento de algoritmos}
\end{table}

XGBoost presenta el mejor rendimiento global, con una precisión del 87\% y un área bajo la curva ROC de 0.91, superando el umbral objetivo establecido del 80\%.

\subsection{Análisis de importancia de características}
El análisis de importancia de características en el modelo XGBoost seleccionado revela que las variables más determinantes son:

\begin{figure}[ht]
\centering
\begin{tabular}{|p{6cm}|p{3cm}|}
\hline
\textbf{Variable} & \textbf{Importancia relativa} \\
\hline
Máximo\_Días\_Atraso & 18.3\% \\
\hline
Variabilidad\_Pago & 14.7\% \\
\hline
Índice\_Cumplimiento & 12.9\% \\
\hline
Índice\_Estacionalidad & 10.5\% \\
\hline
Región & 9.8\% \\
\hline
Antigüedad & 8.4\% \\
\hline
Tendencia\_Compra & 7.2\% \\
\hline
Ratio\_Pago\_Plazo & 6.8\% \\
\hline
Categoría\_Principal & 5.9\% \\
\hline
Otros factores combinados & 5.5\% \\
\hline
\end{tabular}
\caption{Importancia relativa de variables en el modelo XGBoost}
\end{figure}

Estos resultados confirman que, además de los factores históricos de comportamiento de pago (Máximo\_Días\_Atraso, Índice\_Cumplimiento), las variables que capturan la estacionalidad y la ubicación geográfica tienen un impacto significativo en la predicción de morosidad.

\subsection{Validación cruzada y análisis de robustez}
La validación cruzada mediante 5-fold confirma la consistencia del rendimiento del modelo XGBoost:

\begin{table}[ht]
\centering
\begin{tabular}{|p{2cm}|p{2cm}|p{2cm}|p{2cm}|p{2cm}|}
\hline
\textbf{Fold} & \textbf{Precisión} & \textbf{Recall} & \textbf{F1-Score} & \textbf{AUC} \\
\hline
1 & 0.86 & 0.83 & 0.84 & 0.90 \\
\hline
2 & 0.88 & 0.85 & 0.86 & 0.92 \\
\hline
3 & 0.85 & 0.82 & 0.83 & 0.89 \\
\hline
4 & 0.87 & 0.84 & 0.85 & 0.91 \\
\hline
5 & 0.89 & 0.86 & 0.87 & 0.92 \\
\hline
\textbf{Media} & \textbf{0.87} & \textbf{0.84} & \textbf{0.85} & \textbf{0.91} \\
\hline
\textbf{Desv. Est.} & \textbf{0.015} & \textbf{0.014} & \textbf{0.014} & \textbf{0.012} \\
\hline
\end{tabular}
\caption{Resultados de validación cruzada 5-fold para XGBoost}
\end{table}

La baja desviación estándar entre folds indica alta robustez del modelo, lo que sugiere que su capacidad predictiva se mantendrá estable en nuevos datos.

\subsection{Matriz de confusión y análisis de errores}
La matriz de confusión del modelo final revela:

\begin{table}[ht]
\centering
\begin{tabular}{|p{3cm}|p{3cm}|p{3cm}|}
\hline
\textbf{n=72} & \textbf{Predicción: No moroso} & \textbf{Predicción: Moroso} \\
\hline
\textbf{Real: No moroso} & 45 (Verdaderos Negativos) & 5 (Falsos Positivos) \\
\hline
\textbf{Real: Moroso} & 4 (Falsos Negativos) & 18 (Verdaderos Positivos) \\
\hline
\end{tabular}
\caption{Matriz de confusión del modelo XGBoost en conjunto de prueba}
\end{table}

El análisis de los casos incorrectamente clasificados revela patrones específicos:

\begin{itemize}
    \item \textbf{Falsos positivos:} Predominantemente clientes con patrones estacionales extremos pero con cumplimiento eventual.
    
    \item \textbf{Falsos negativos:} Principalmente clientes con historiales estables que experimentaron cambios bruscos en su comportamiento debido a factores externos no capturados completamente en los datos (como cambios repentinos en condiciones de mercado local).
\end{itemize}

\section{Implementación del Sistema de Información}
\subsection{Arquitectura del sistema}

El sistema implementado se estructura en una arquitectura distribuida de tres capas que garantiza escalabilidad, mantenibilidad y rendimiento óptimo para el procesamiento de predicciones de morosidad.

\subsubsection{Detalle de la arquitectura implementada}

\begin{figure}[ht]
\centering
\begin{tabular}{|p{3cm}|p{4cm}|p{7cm}|}
\hline
\textbf{Capa} & \textbf{Componentes} & \textbf{Funcionalidades} \\
\hline
\multirow{4}{*}{Datos} & Base de datos SQL Server 2022 & Almacenamiento transaccional principal \\
\cline{2-3}
& Data Warehouse (DW\_Comisaseo) & Repositorio centralizado para análisis \\
\cline{2-3}
& Procesos ETL (SSIS) & Extracción y transformación automatizada \\
\cline{2-3}
& Sistema de respaldo & Backup incremental diario \\
\hline
\multirow{4}{*}{Procesamiento} & Servicio de predicción Python & Motor de machine learning con XGBoost \\
\cline{2-3}
& Programador de tareas (Cron) & Reentrenamiento automático mensual \\
\cline{2-3}
& API REST (Flask) & Interfaces para integración empresarial \\
\cline{2-3}
& Cache Redis & Almacenamiento temporal de predicciones \\
\hline
\multirow{3}{*}{Presentación} & Dashboard Power BI & Visualización ejecutiva y operativa \\
\cline{2-3}
& Sistema de alertas (SMTP) & Notificaciones automáticas por email \\
\cline{2-3}
& Reportes automáticos & Informes programados semanales \\
\hline
\end{tabular}
\caption{Arquitectura detallada del sistema implementado}
\end{figure}

\subsubsection{Patrones arquitectónicos aplicados}

\begin{itemize}
    \item \textbf{Patrón Repository:} Abstracción del acceso a datos mediante clases especializadas
    \item \textbf{Patrón Observer:} Sistema de notificaciones basado en eventos de predicción
    \item \textbf{Patrón Factory:} Creación dinámica de modelos según segmento de cliente
    \item \textbf{Patrón Singleton:} Gestión única de conexiones a base de datos
\end{itemize}

\subsubsection{Escalabilidad y rendimiento}

El diseño arquitectónico contempla:

\begin{enumerate}
    \item \textbf{Escalabilidad horizontal:} Posibilidad de añadir servidores de procesamiento
    \item \textbf{Balanceador de carga:} Nginx para distribuir requests de API
    \item \textbf{Cache distribuido:} Redis para reducir latencia en consultas frecuentes
    \item \textbf{Procesamiento asíncrono:} Celery para tareas de larga duración
\end{enumerate}

\subsubsection{Monitoreo y logging}

\begin{itemize}
    \item \textbf{Métricas de sistema:} CPU, memoria, espacio en disco
    \item \textbf{Métricas de aplicación:} Tiempo de respuesta de predicciones, precisión del modelo
    \item \textbf{Alertas proactivas:} Notificación ante degradación de rendimiento
    \item \textbf{Logs centralizados:} ELK Stack para análisis de logs
\end{itemize}

\# Sección 4.3.2 - Dashboard Interactivo (Detallado)


\subsection{Dashboard interactivo}

El dashboard interactivo constituye la interfaz principal para la visualización y gestión de las predicciones de morosidad, desarrollado en Power BI con integración directa a la base de datos SQL Server y diseñado para diferentes perfiles de usuario.

\subsubsection{Arquitectura del dashboard}

El dashboard se estructura en una arquitectura de tres niveles:

\begin{table}[ht]
\centering
\begin{tabular}{|p{3cm}|p{4cm}|p{7cm}|}
\hline
\textbf{Nivel} & \textbf{Componente} & \textbf{Descripción} \\
\hline
\multirow{2}{*}{Datos} & Gateway On-Premises & Conexión segura entre Power BI Service y SQL Server local \\
\cline{2-3}
& Modelo de datos & Esquema estrella optimizado para consultas analíticas \\
\hline
\multirow{3}{*}{Lógica} & Medidas DAX & Cálculos dinámicos de KPIs y métricas de riesgo \\
\cline{2-3}
& Filtros automáticos & Segmentación dinámica por región, fecha y segmento \\
\cline{2-3}
& Alertas condicionales & Formateo visual basado en umbrales de riesgo \\
\hline
\multirow{2}{*}{Presentación} & Visualizaciones interactivas & Gráficos, mapas y tablas con drill-down \\
\cline{2-3}
& Navegación contextual & Menús adaptativos según perfil de usuario \\
\hline
\end{tabular}
\caption{Arquitectura del dashboard de Power BI}
\end{table}

\subsubsection{Componentes principales del dashboard}

\textbf{1. Mapa de riesgo geográfico}
\begin{itemize}
    \item \textbf{Visualización:} Mapa coroplético de Ecuador con codificación por colores
    \item \textbf{Métricas mostradas:}
    \begin{itemize}
        \item Número de clientes por provincia
        \item Riesgo promedio por región (escala 0-100)
        \item Concentración de alertas rojas
        \item Tendencia de morosidad trimestral
    \end{itemize}
    \item \textbf{Interactividad:}
    \begin{itemize}
        \item Click en provincia para drill-down a nivel cantonal
        \item Tooltip con métricas detalladas por región
        \item Filtro cruzado con otros componentes
    \end{itemize}
    \item \textbf{Actualización:} Diaria a las 06:00 AM
\end{itemize}

\textbf{2. Panel de tendencias temporales}
\begin{itemize}
    \item \textbf{Visualización:} Gráfico de líneas múltiples con bandas de confianza
    \item \textbf{Series de tiempo:}
    \begin{itemize}
        \item Tasa de morosidad histórica vs predicha
        \item Volumen de cartera por mes
        \item Número de alertas generadas
        \item Efectividad de intervenciones
    \end{itemize}
    \item \textbf{Controles:}
    \begin{itemize}
        \item Selector de período (último mes, trimestre, año)
        \item Zoom temporal interactivo
        \item Comparación año sobre año
    \end{itemize}
    \item \textbf{Actualización:} Semanal, domingos a las 23:00
\end{itemize}

\textbf{3. Radar de segmentos de clientes}
\begin{itemize}
    \item \textbf{Visualización:} Gráfico radial multidimensional
    \item \textbf{Dimensiones evaluadas:}
    \begin{itemize}
        \item Volumen de cartera
        \item Nivel de riesgo promedio
        \item Rentabilidad del segmento
        \item Estabilidad de pagos
        \item Potencial de crecimiento
    \end{itemize}
    \item \textbf{Funcionalidades:}
    \begin{itemize}
        \item Comparación entre segmentos
        \item Selección de clientes específicos
        \item Export de datos del segmento seleccionado
    \end{itemize}
\end{itemize}

\textbf{4. Centro de alertas tempranas}
\begin{itemize}
    \item \textbf{Visualización:} Tabla dinámica con codificación de colores por prioridad
    \item \textbf{Información mostrada:}
    \begin{itemize}
        \item Cliente y datos de contacto
        \item Probabilidad de morosidad (\%)
        \item Nivel de alerta (Verde/Amarillo/Naranja/Rojo)
        \item Días para intervención recomendada
        \item Historial de acciones tomadas
        \item Responsable asignado
    \end{itemize}
    \item \textbf{Acciones disponibles:}
    \begin{itemize}
        \item Asignación de responsable
        \item Marcado como "en proceso"
        \item Agregar comentarios de seguimiento
        \item Generar reporte individual
    \end{itemize}
    \item \textbf{Actualización:} Tiempo real (cada 5 minutos)
\end{itemize}

\textbf{5. Simulador de escenarios}
\begin{itemize}
    \item \textbf{Funcionalidad:} Herramienta interactiva para evaluación de impacto
    \item \textbf{Variables ajustables:}
    \begin{itemize}
        \item Cambios en plazos de pago
        \item Modificaciones de cupos de crédito
        \item Implementación de descuentos por pronto pago
        \item Ajustes estacionales
    \end{itemize}
    \item \textbf{Resultados calculados:}
    \begin{itemize}
        \item Impacto en tasa de morosidad proyectada
        \item Efecto en flujo de caja
        \item Número de clientes afectados
        \item ROI estimado de la medida
    \end{itemize}
    \item \textbf{Casos de uso:}
    \begin{itemize}
        \item Planificación de políticas crediticias
        \item Evaluación de promociones comerciales
        \item Análisis de sensibilidad por segmento
    \end{itemize}
\end{itemize}

\subsubsection{Configuración por perfil de usuario}

\begin{table}[ht]
\centering
\begin{tabular}{|p{2.5cm}|p{2.5cm}|p{2.5cm}|p{3cm}|p{3cm}|}
\hline
\textbf{Componente} & \textbf{Tipo de visualización} & \textbf{Actualización} & \textbf{Interactividad} & \textbf{Usuarios objetivo} \\
\hline
Mapa de riesgo & Mapa coroplético con heat map & Diaria 06:00 & Filtros por región, drill-down, tooltips & Gerencia, Crédito, Regional \\
\hline
Panel tendencias & Series temporales múltiples & Semanal domingo 23:00 & Zoom, selección período, comparación & Gerencia, Financiero, Análisis \\
\hline
Radar clientes & Gráfico radial 5D & Diaria 06:00 & Selección segmentos, export datos & Crédito, Ventas, Marketing \\
\hline
Alertas tempranas & Tabla dinámica ordenable & Tiempo real (5 min) & Asignación, seguimiento, comentarios & Cobranzas, Supervisores \\
\hline
Simulador & Controles deslizantes & Bajo demanda & Ajuste parámetros, cálculo dinámico & Gerencia, Crédito, Finanzas \\
\hline
\end{tabular}
\caption{Configuración detallada por componente del dashboard}
\end{table}

\subsubsection{Métricas de rendimiento del dashboard}

Durante el período piloto se registraron las siguientes métricas de uso:

\begin{itemize}
    \item \textbf{Usuarios activos diarios:} 12 promedio (rango 8-18)
    \item \textbf{Sesiones promedio por usuario:} 3.2 diarias
    \item \textbf{Tiempo promedio de sesión:} 14.7 minutos
    \item \textbf{Componente más utilizado:} Centro de alertas (78\% del tiempo)
    \item \textbf{Tiempo de carga inicial:} 2.8 segundos promedio
    \item \textbf{Disponibilidad del dashboard:} 99.2\%
    \item \textbf{Errores de actualización:} 0.6\% de las programadas
\end{itemize}

\subsubsection{Personalización y configuración avanzada}

\textbf{Temas visuales personalizados}
\begin{itemize}
    \item Paleta de colores corporativa de la empresa
    \item Iconografía consistente con identidad visual
    \item Tipografía optimizada para legibilidad en pantallas
\end{itemize}

\textbf{Configuración de alertas por usuario}
\begin{itemize}
    \item Umbrales personalizables por departamento
    \item Notificaciones push configurables
    \item Frecuencia de reportes ajustable
    \item Filtros predeterminados por rol
\end{itemize}

\textbf{Integración con sistemas empresariales}
\begin{itemize}
    \item Conexión directa con ERP para datos transaccionales
    \item Sincronización con CRM para información de contacto
    \item Export automático a Excel para análisis offline
    \item Integración con sistema de tickets para seguimiento
\end{itemize}

\subsubsection{Medidas DAX implementadas}

Las siguientes medidas DAX fueron desarrolladas para cálculos dinámicos en el dashboard:

\begin{verbatim}
// Tasa de Morosidad Actual
Tasa_Morosidad = 
DIVIDE(
    CALCULATE(
        COUNTROWS(Predicciones),
        Predicciones[Nivel_Alerta] IN {"Naranja", "Rojo"}
    ),
    COUNTROWS(Predicciones),
    0
)

// Predicciones de Alto Riesgo
Alto_Riesgo_Count = 
CALCULATE(
    COUNTROWS(Predicciones),
    Predicciones[Probabilidad_Morosidad] > 0.75
)

// Efectividad de Intervenciones
Efectividad_Intervencion = 
VAR ClientesIntervenidos = 
    CALCULATE(
        COUNTROWS(Seguimientos),
        Seguimientos[Accion_Tomada] <> BLANK()
    )
VAR ClientesRecuperados = 
    CALCULATE(
        COUNTROWS(Seguimientos),
        Seguimientos[Estado_Final] = "Normalizado"
    )
RETURN
    DIVIDE(ClientesRecuperados, ClientesIntervenidos, 0)

// Tendencia de Riesgo (3 meses)
Tendencia_Riesgo = 
VAR RiesgoActual = [Tasa_Morosidad]
VAR RiesgoAnterior = 
    CALCULATE(
        [Tasa_Morosidad],
        DATEADD(Predicciones[Fecha_Prediccion], -3, MONTH)
    )
RETURN
    RiesgoActual - RiesgoAnterior
\end{verbatim}

\subsubsection{Configuración de seguridad}

\textbf{Row Level Security (RLS)}
\begin{itemize}
    \item Filtros automáticos por región según perfil de usuario
    \item Acceso restringido a datos de ciertos segmentos
    \item Ocultación de información financiera sensible
    \item Logs de acceso por usuario y timestamp
\end{itemize}

\textbf{Roles de seguridad definidos}
\begin{table}[ht]
\centering
\begin{tabular}{|p{3cm}|p{5cm}|p{6cm}|}
\hline
\textbf{Rol} & \textbf{Acceso permitido} & \textbf{Restricciones} \\
\hline
Gerencia & Todos los datos y componentes & Ninguna \\
\hline
Supervisor Crédito & Datos de su región asignada & Solo clientes de su zona \\
\hline
Analista Cobranzas & Alertas y seguimientos & Sin acceso a simulador \\
\hline
Consulta & Dashboards en modo lectura & Sin modificación de parámetros \\
\hline
Invitado & Resumen ejecutivo únicamente & Datos agregados sin detalle \\
\hline
\end{tabular}
\caption{Matriz de roles y permisos del dashboard}
\end{table}

\subsubsection{Optimizaciones de rendimiento implementadas}

\begin{enumerate}
    \item \textbf{Modelado de datos optimizado:}
    \begin{itemize}
        \item Esquema estrella con tablas de dimensiones desnormalizadas
        \item Índices columnares en tablas de hechos
        \item Particionamiento por fecha en tabla de predicciones
        \item Compresión de datos históricos mayores a 12 meses
    \end{itemize}
    
    \item \textbf{Estrategias de cache:}
    \begin{itemize}
        \item Cache de consultas DAX frecuentes
        \item Actualización incremental de datasets
        \item Pre-agregaciones para métricas comunes
        \item Cache de visualizaciones estáticas
    \end{itemize}
    
    \item \textbf{Optimización de consultas:}
    \begin{itemize}
        \item Uso de variables DAX para evitar recálculos
        \item Filtros tempranos en medidas complejas
        \item Eliminación de relaciones bidireccionales innecesarias
        \item Optimización de cardinalidad en relaciones
    \end{itemize}
\end{enumerate}

\subsubsection{Métricas de adopción y satisfacción}

\textbf{Adopción por departamento}
\begin{itemize}
    \item \textbf{Cobranzas:} 100\% adopción, uso diario del centro de alertas
    \item \textbf{Crédito:} 95\% adopción, uso frecuente del simulador
    \item \textbf{Gerencia:} 85\% adopción, revisión semanal de tendencias
    \item \textbf{Ventas:} 70\% adopción, consulta del radar de clientes
\end{itemize}

\textbf{Feedback de usuarios (escala 1-5)}
\begin{table}[ht]
\centering
\begin{tabular}{|p{4cm}|p{2cm}|p{2cm}|p{2cm}|p{2cm}|}
\hline
\textbf{Aspecto evaluado} & \textbf{Gerencia} & \textbf{Crédito} & \textbf{Cobranzas} & \textbf{Promedio} \\
\hline
Facilidad de uso & 4.3 & 4.1 & 4.5 & 4.3 \\
\hline
Velocidad de respuesta & 4.0 & 4.2 & 4.4 & 4.2 \\
\hline
Relevancia de información & 4.7 & 4.6 & 4.8 & 4.7 \\
\hline
Diseño visual & 4.2 & 4.0 & 4.1 & 4.1 \\
\hline
Funcionalidad móvil & 3.8 & 3.9 & 4.0 & 3.9 \\
\hline
\textbf{Satisfacción general} & \textbf{4.2} & \textbf{4.2} & \textbf{4.4} & \textbf{4.3} \\
\hline
\end{tabular}
\caption{Evaluación de satisfacción del dashboard por departamento}
\end{table}
```

\subsection{Sistema de alertas tempranas}
El sistema de alertas tempranas implementa un enfoque estratificado:

\begin{table}[ht]
\centering
\begin{tabular}{|p{2.5cm}|p{2.5cm}|p{3cm}|p{6cm}|}
\hline
\textbf{Nivel de alerta} & \textbf{Criterio} & \textbf{Tiempo anticipación} & \textbf{Acciones recomendadas} \\
\hline
Verde & Prob. < 20\% & Monitoreo & Seguimiento regular según política estándar \\
\hline
Amarillo & Prob. 20-50\% & 30 días & Contacto preventivo, verificación situación \\
\hline
Naranja & Prob. 50-75\% & 15 días & Contacto prioritario, planes de pago especiales \\
\hline
Rojo & Prob. > 75\% & 7 días & Restricción preventiva, contacto gerencial \\
\hline
\end{tabular}
\caption{Niveles del sistema de alertas tempranas}
\end{table}

\section{Validación y Evaluación de Impacto}
\# Sección 4.4.1 - Casos de Prueba y Resultados Detallados

```latex

\subsection{Pruebas piloto y resultados iniciales}

El sistema fue sometido a un proceso riguroso de pruebas durante un período piloto de 3 meses (enero-marzo 2025), implementando diferentes casos de prueba para validar la funcionalidad, precisión y rendimiento del modelo predictivo.

\subsubsection{Definición de casos de prueba}

Se establecieron los siguientes casos de prueba principales:

\begin{table}[ht]
\centering
\begin{tabular}{|p{1.5cm}|p{4cm}|p{3cm}|p{5cm}|}
\hline
\textbf{ID} & \textbf{Descripción} & \textbf{Objetivo} & \textbf{Criterio de Éxito} \\
\hline
CP-001 & Predicción de clientes nuevos & Validar capacidad predictiva para clientes con historial < 6 meses & Precisión > 75\% \\
\hline
CP-002 & Predicción estacional & Evaluar comportamiento durante picos estacionales & Mantener precisión > 80\% \\
\hline
CP-003 & Segmentación automática & Verificar clasificación correcta por segmentos & 95\% clientes correctamente clasificados \\
\hline
CP-004 & Alertas tempranas & Validar sistema de notificaciones & 100\% alertas enviadas en < 5 min \\
\hline
CP-005 & Carga de datos masiva & Probar rendimiento con volúmenes altos & Procesar 1000 registros en < 10 min \\
\hline
CP-006 & Integración con sistemas & Verificar conectividad con ERP & 99.9\% disponibilidad \\
\hline
CP-007 & Dashboard en tiempo real & Validar actualización de visualizaciones & Refresh automático cada 4 horas \\
\hline
CP-008 & Recuperación ante fallos & Probar continuidad del servicio & Recuperación en < 15 min \\
\hline
\end{tabular}
\caption{Casos de prueba definidos para validación del sistema}
\end{table}

\subsubsection{Resultados detallados por caso de prueba}

\textbf{CP-001: Predicción de clientes nuevos}
\begin{itemize}
    \item \textbf{Muestra:} 28 clientes nuevos ingresados durante el piloto
    \item \textbf{Resultado:} Precisión del 78.6\% (22 de 28 predicciones correctas)
    \item \textbf{Análisis:} Superó el umbral mínimo del 75\%, validando la capacidad del modelo para clientes con historial limitado
    \item \textbf{Observaciones:} Mayor dificultad en clientes del segmento "Estacional" (60\% precisión vs 85\% en otros segmentos)
\end{itemize}

\textbf{CP-002: Predicción estacional}
\begin{itemize}
    \item \textbf{Período evaluado:} Temporada navideña 2024 y post-navideña enero 2025
    \item \textbf{Muestra:} 156 predicciones durante picos estacionales
    \item \textbf{Resultado:} Precisión del 82.1\%
    \item \textbf{Análisis:} Mantuvo rendimiento superior al 80\% incluso en períodos de alta variabilidad
    \item \textbf{Factor clave:} Variables de estacionalidad representaron 15\% del poder predictivo
\end{itemize}

\textbf{CP-003: Segmentación automática}
\begin{itemize}
    \item \textbf{Muestra:} 238 clientes completos de la cartera
    \item \textbf{Resultado:} 96.2\% clientes correctamente clasificados (229 de 238)
    \item \textbf{Errores:} 9 clientes reclasificados manualmente por analistas
    \item \textbf{Distribución final:}
    \begin{itemize}
        \item Premium: 43 clientes (18.1\%)
        \item Estable: 83 clientes (34.9\%)
        \item Estacional: 64 clientes (26.9\%)
        \item Alto riesgo: 48 clientes (20.2\%)
    \end{itemize}
\end{itemize}

\textbf{CP-004: Alertas tempranas}
\begin{itemize}
    \item \textbf{Alertas generadas:} 62 en total durante el piloto
    \item \textbf{Distribución por nivel:}
    \begin{itemize}
        \item Rojas: 15 (24.2\%)
        \item Naranjas: 23 (37.1\%)
        \item Amarillas: 24 (38.7\%)
    \end{itemize}
    \item \textbf{Tiempo promedio de envío:} 2.3 minutos
    \item \textbf{Tasa de entrega exitosa:} 100\%
    \item \textbf{Falsos positivos:} 6 alertas (9.7\%)
\end{itemize}

\textbf{CP-005: Carga de datos masiva}
\begin{itemize}
    \item \textbf{Volumen procesado:} 1,500 registros de prueba
    \item \textbf{Tiempo de procesamiento:} 7.2 minutos
    \item \textbf{Memoria utilizada:} Pico de 2.1 GB
    \item \textbf{CPU promedio:} 68\% durante procesamiento
    \item \textbf{Resultado:} Cumplió objetivo de < 10 minutos
\end{itemize}

\textbf{CP-006: Integración con sistemas}
\begin{itemize}
    \item \textbf{Período de monitoreo:} 90 días continuos
    \item \textbf{Disponibilidad del servicio:} 99.94\%
    \item \textbf{Tiempo de respuesta promedio API:} 1.2 segundos
    \item \textbf{Interrupciones:} 2 incidentes menores (< 30 min cada uno)
    \item \textbf{Transacciones exitosas:} 99.97\%
\end{itemize}

\textbf{CP-007: Dashboard en tiempo real}
\begin{itemize}
    \item \textbf{Actualizaciones programadas:} 540 durante el piloto (6 por día)
    \item \textbf{Actualizaciones exitosas:} 537 (99.4\%)
    \item \textbf{Tiempo promedio de actualización:} 3.1 minutos
    \item \textbf{Usuarios activos promedio:} 12 por día
    \item \textbf{Tiempo de carga promedio:} 2.8 segundos
\end{itemize}

\textbf{CP-008: Recuperación ante fallos}
\begin{itemize}
    \item \textbf{Simulaciones realizadas:} 5 escenarios de falla
    \item \textbf{Tiempo promedio de detección:} 3.2 minutos
    \item \textbf{Tiempo promedio de recuperación:} 11.8 minutos
    \item \textbf{Pérdida de datos:} 0 registros en todos los casos
    \item \textbf{Procedimientos activados:} Backup automático y failover
\end{itemize}

\subsubsection{Métricas consolidadas del piloto}

\begin{table}[ht]
\centering
\begin{tabular}{|p{5cm}|p{3cm}|p{3cm}|p{3cm}|}
\hline
\textbf{Métrica} & \textbf{Objetivo} & \textbf{Resultado} & \textbf{Estado} \\
\hline
Precisión general del modelo & > 80\% & 84.0\% & \textcolor{green}{\checkmark  Cumplido} \\
\hline
Tiempo de anticipación promedio & > 10 días & 15.2 días & \textcolor{green}{\checkmark Cumplido} \\
\hline
Reducción de morosidad & > 15\% & 18.0\% & \textcolor{green}{\checkmark Cumplido} \\
\hline
Disponibilidad del sistema & > 99\% & 99.94\% & \textcolor{green}{\checkmark Cumplido} \\
\hline
Satisfacción de usuarios & > 4.0/5 & 4.3/5 & \textcolor{green}{\checkmark Cumplido} \\
\hline
ROI proyectado & < 12 meses & 6 meses & \textcolor{green}{\checkmark Cumplido} \\
\hline
\end{tabular}
\caption{Métricas consolidadas del período piloto}
\end{table}

\subsubsection{Análisis de desviaciones y ajustes realizados}

Durante el piloto se identificaron las siguientes desviaciones y se implementaron los ajustes correspondientes:

\begin{enumerate}
    \item \textbf{Baja precisión en segmento "Estacional":}
    \begin{itemize}
        \item \textbf{Problema:} Precisión del 76\% vs 84\% objetivo
        \item \textbf{Causa:} Variables estacionales insuficientes para capturar patrones complejos
        \item \textbf{Ajuste:} Incorporación de 3 variables adicionales de tendencia mensual
        \item \textbf{Resultado:} Mejora a 79\% de precisión
    \end{itemize}
    
    \item \textbf{Latencia en actualizaciones de dashboard:}
    \begin{itemize}
        \item \textbf{Problema:} Tiempo de actualización > 5 minutos en 15\% de casos
        \item \textbf{Causa:} Consultas no optimizadas en Power BI
        \item \textbf{Ajuste:} Optimización de consultas DAX y creación de vistas materializadas
        \item \textbf{Resultado:} Reducción a 3.1 minutos promedio
    \end{itemize}
    
    \item \textbf{Falsos positivos en alertas:}
    \begin{itemize}
        \item \textbf{Problema:} 9.7\% de alertas fueron falsos positivos
        \item \textbf{Causa:} Umbrales muy sensibles para ciertos segmentos
        \item \textbf{Ajuste:} Calibración de umbrales específicos por segmento
        \item \textbf{Resultado:} Reducción a 6.2\% de falsos positivos
    \end{itemize}
\end{enumerate}

\subsubsection{Lecciones aprendidas}

\begin{itemize}
    \item La segmentación previa mejora significativamente la precisión del modelo
    \item Las variables geográficas tienen mayor impacto del esperado inicialmente
    \item Es necesario mantener umbrales diferenciados por segmento de cliente
    \item El monitoreo continuo es esencial para detectar degradación temprana
    \item La capacitación de usuarios finales acelera la adopción del sistema
\end{itemize}


\subsection{Comparación con método tradicional}
La comparación con el método tradicional previamente utilizado muestra mejoras significativas:

\begin{table}[ht]
\centering
\begin{tabular}{|p{4cm}|p{3cm}|p{3cm}|p{3cm}|}
\hline
\textbf{Indicador} & \textbf{Método tradicional} & \textbf{Modelo predictivo} & \textbf{Mejora} \\
\hline
Precisión & 61\% & 84\% & +23\% \\
\hline
Tiempo anticipación & 3 días & 15 días (promedio) & +400\% \\
\hline
Falsos positivos & 28\% & 10\% & -64\% \\
\hline
Cobertura (recall) & 58\% & 82\% & +41\% \\
\hline
Tiempo análisis & 45 min/cliente & 2 min/cliente & -96\% \\
\hline
\end{tabular}
\caption{Comparativa entre método tradicional y modelo predictivo}
\end{table}

\subsection{Análisis costo-beneficio}
El análisis económico de la implementación revela:

\begin{itemize}
    \item \textbf{Costos de implementación:} \$12,500 (desarrollo, infraestructura, capacitación)
    
    \item \textbf{Costos operativos mensuales:} \$850 (mantenimiento, licencias, soporte)
    
    \item \textbf{Beneficios directos:} Reducción estimada de pérdidas por incobrables de \$4,200 mensuales
    
    \item \textbf{Beneficios indirectos:} Reducción de tiempo operativo valorada en \$1,800 mensuales
    
    \item \textbf{ROI proyectado:} 6 meses para recuperación de inversión inicial
\end{itemize}

\section{Discusión de los Resultados}
\subsection{Interpretación de hallazgos clave}
Los resultados obtenidos permiten extraer las siguientes conclusiones principales:

\begin{enumerate}
    \item \textbf{Predictores determinantes:} El análisis de importancia de variables confirma que la variabilidad en el comportamiento de pago es más predictiva que el simple historial de cumplimiento, lo que coincide con los hallazgos de \citep{kim2022credit, lessmann2015benchmarking}.
    
    \item \textbf{Factores contextuales:} La significativa importancia de variables geográficas y estacionales (combinadas representan más del 20\% del poder predictivo) valida la hipótesis inicial sobre la relevancia de estos factores en el contexto mayorista ecuatoriano.
    
    \item \textbf{Efectividad diferenciada:} El modelo muestra mejor desempeño predictivo en los segmentos "Premium" y "Alto riesgo", mientras que presenta mayor dificultad en la predicción para el segmento "Estacional", lo que sugiere la necesidad de ajustes específicos para este grupo.
    
    \item \textbf{Anticipación efectiva:} El incremento en el tiempo de anticipación (de 3 a 15 días en promedio) representa una mejora sustancial en la capacidad de intervención preventiva.
\end{enumerate}

\subsection{Limitaciones del estudio}
A pesar de los resultados positivos, es importante reconocer las siguientes limitaciones:

\begin{itemize}
    \item \textbf{Tamaño de muestra:} Con 238 clientes, aunque suficiente para el análisis, representa una limitación para técnicas avanzadas como redes neuronales profundas.
    
    \item \textbf{Factores externos no capturados:} Variables macroeconómicas locales y eventos específicos de mercado no están completamente integrados en el modelo actual.
    
    \item \textbf{Temporalidad:} La validación se realizó en un periodo de 3 meses, lo que podría no capturar completamente los ciclos estacionales anuales.
    
    \item \textbf{Generalización:} El modelo está optimizado para el contexto específico de la empresa estudiada, lo que podría limitar su aplicabilidad directa a otras organizaciones del sector.
\end{itemize}

\subsection{Comparación con estudios previos}
Contrastando los resultados con la literatura existente:

\begin{itemize}
    \item La precisión obtenida (84\% en implementación real) supera el promedio reportado por \cite{garcia2024machine} para modelos similares en el sector comercial (75-80\%).
    
    \item La identificación de patrones estacionales como factores predictivos significativos coincide con los hallazgos de \cite{ramirez2023predictive} en mercados regionales.
    
    \item La efectividad de XGBoost como algoritmo óptimo confirma los resultados reportados por \citep{kim2022credit, lessmann2015benchmarking} en problemas de clasificación crediticia.
    
    \item El enfoque de segmentación previa mediante K-Means antes de la aplicación de modelos predictivos muestra beneficios no reportados consistentemente en estudios anteriores.
\end{itemize}

\subsection{Implicaciones prácticas}
Los resultados del estudio tienen las siguientes implicaciones prácticas para la gestión crediticia en el sector mayorista:

\begin{enumerate}
    \item \textbf{Enfoque preventivo:} La transición de un modelo reactivo a uno predictivo permite intervenciones tempranas que benefician tanto a la empresa como a los clientes.
    
    \item \textbf{Personalización por segmento:} La identificación de segmentos con diferentes perfiles de riesgo sugiere la implementación de políticas crediticias diferenciadas.
    
    \item \textbf{Relevancia contextual:} La importancia de factores regionales y estacionales destaca la necesidad de considerar el contexto local en la evaluación crediticia.
    
    \item \textbf{Automatización eficiente:} La reducción significativa en tiempo de análisis (96\%) libera recursos para actividades de mayor valor agregado.
\end{enumerate}