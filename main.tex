\documentclass[12pt,a4paper]{report}

% ======================================
% CONFIGURACIÓN DE FUENTES
% ======================================

% OPCIÓN 1: Times New Roman (Recomendado para tesis tradicional)
%\usepackage{mathptmx} % Times New Roman para texto y matemáticas
%\usepackage[T1]{fontenc}

% OPCIÓN 2: Arial (Descomenta estas líneas si prefieres Arial)
% \usepackage{helvet}
% \renewcommand{\familydefault}{\sfdefault}
% \usepackage[T1]{fontenc}

% OPCIÓN 3: Calibri (Requiere XeLaTeX o LuaLaTeX)
% \usepackage{fontspec}
% \setmainfont{Calibri}



% Configuración de codificación (OBLIGATORIO AL INICIO)
\usepackage[utf8]{inputenc}
\usepackage[T1]{fontenc}
\usepackage[spanish]{babel}
\usepackage{textcomp}
\usepackage{afterpage}
% Paquetes para tablas (IMPORTANTE: incluir multirow)
\usepackage{array}
\usepackage{tabularx}
\usepackage{longtable}
\usepackage{booktabs}
\usepackage{multirow}     % ← ESTE ES EL PAQUETE QUE FALTA
\usepackage{multicol}
\usepackage{colortbl}
\usepackage{hyperref}
\usepackage{natbib}
\usepackage{setspace}

% Paquetes para código fuente
\usepackage{listings}
\usepackage{verbatim}
\usepackage{fancyvrb}

% Paquetes para gráficos y figuras
\usepackage{graphicx}
\usepackage{float}
\usepackage{subfig}
\usepackage{caption}
\usepackage{placeins}
% Paquetes matemáticos
\usepackage{amsmath}
\usepackage{amsfonts}
\usepackage{amssymb}
\usepackage{mathtools}

% Paquetes para diseño y formato
\usepackage{geometry}
\usepackage{setspace}
\usepackage{titlesec}
\usepackage{fancyhdr}
\usepackage{xcolor}

% Configuración de márgenes
\geometry{left=3cm,right=2.5cm,top=2.5cm,bottom=2.5cm}
% Configuración para código SQL/Python
\lstset{
    inputencoding=utf8,
    extendedchars=true,
    basicstyle=\ttfamily\footnotesize,
    keywordstyle=\color{blue}\bfseries,
    commentstyle=\color{green!50!black},
    stringstyle=\color{red},
    numbers=left,
    numberstyle=\tiny\color{gray},
    stepnumber=1,
    numbersep=8pt,
    showstringspaces=false,
    breaklines=true,
    frame=single,
    backgroundcolor=\color{gray!10},
    literate={á}{{\'a}}1 {é}{{\'e}}1 {í}{{\'i}}1 {ó}{{\'o}}1 {ú}{{\'u}}1
             {Á}{{\'A}}1 {É}{{\'E}}1 {Í}{{\'I}}1 {Ó}{{\'O}}1 {Ú}{{\'U}}1
             {ñ}{{\~n}}1 {Ñ}{{\~N}}1 {ü}{{\"u}}1 {Ü}{{\"U}}1
}
% Configuración de listings
\lstset{
    inputencoding=utf8,
    extendedchars=true,
    basicstyle=\ttfamily\footnotesize,
    breaklines=true,
    frame=single
}
% Configuración para HTML
\lstdefinelanguage{HTML5}{
    language=html,
    sensitive=true,
    alsoletter={<>=-},
    morecomment=[s]{<!--}{-->},
    tag=[s],
    morestring=[b]",
    morestring=[b]',
    morekeywords={html,head,body,div,span,p,h1,h2,h3,h4,h5,h6,
                 title,meta,link,style,script,header,footer,nav,
                 section,article,aside,main,form,input,button,
                 table,tr,td,th,ul,ol,li,a,img,br,hr,canvas,svg}
}

\lstset{
    language=HTML5,
    basicstyle=\ttfamily\footnotesize,
    keywordstyle=\color{blue}\bfseries,
    commentstyle=\color{green!50!black}\itshape,
    stringstyle=\color{red},
    numberstyle=\tiny\color{gray},
    numbers=left,
    numbersep=5pt,
    breaklines=true,
    breakatwhitespace=false,
    frame=single,
    backgroundcolor=\color{gray!5},
    showspaces=false,
    showstringspaces=false,
    showtabs=false,
    tabsize=2,
    captionpos=b
}
% ======================================
% CONFIGURACIÓN PARA TÍTULOS
% ======================================
\usepackage{titlesec}
\titleformat{\chapter}{\normalfont\huge\bfseries}{\thechapter.}{20pt}{\huge}
\titlespacing*{\chapter}{0pt}{-50pt}{40pt}

% ======================================
% CONFIGURACIÓN DE URLs
% ======================================
\hypersetup{
    colorlinks=true,
    linkcolor=blue,
    filecolor=magenta,
    urlcolor=cyan,
    pdftitle={Desarrollo de Modelo predictivo basado en machine learning para análisis de morosidad de cartera},
    pdfpagemode=FullScreen,
}

% ======================================
% INTERLINEADO
% ======================================
\onehalfspacing

% ======================================
% CONFIGURACIONES ADICIONALES PARA TESIS BI
% ======================================
% Para mejorar las tablas de datos
\usepackage{longtable} % Para tablas largas
\usepackage{rotating}  % Para rotar tablas grandes
\usepackage{pdflscape} % Para páginas en landscape

% Para mejor formato de números
\usepackage{siunitx}
\sisetup{
    output-decimal-marker = {,},
    group-separator = {.},
    group-minimum-digits = 4
}

\begin{document}

% ======================================
% PÁGINAS PRELIMINARES
% ======================================
% Portada
\begin{titlepage}
    \begin{center}
        \vspace*{0.5cm}
        
        \Huge
        \textbf{UNIVERSIDAD TECNOLÓGICA ECOTEC}
        
        \vspace{1cm}
        
        \Large
        MAESTRÍA EN SISTEMAS DE INFORMACIÓN CON MENCIÓN EN INTELIGENCIA DE NEGOCIOS
        
        \vspace{0.5 cm}
        
        \includegraphics[width=0.4\textwidth]{images/logo.jpg}
        
        \vspace{0.5 cm}
        
        \Huge
        \textbf{Desarrollo de Modelo predictivo basado en machine learning para la identificación temprana de riesgo de morosidad en carteras crediticias}
        
        \vspace{0.5 cm}
        
        \Large
        \textbf{PROYECTO DE DESARROLLO}
        
        \vspace{0.5 cm}
        
        \Large
        \textbf{Autor:}\\
        Orlando Daniel Zambrano Zúñiga\\
        
        \vspace{0.5cm}
        
        \textbf{Tutor:}\\
        Magíster Christian Merchán\\
        
        \vfill
        
        \Large
        Guayaquil, Ecuador\\
        2025
    \end{center}
\end{titlepage}

% Declaración de autoría original
\chapter*{Declaración de Autoría Original}
\thispagestyle{empty}

Por la presente declaro que soy el autor de este Trabajo de Titulación titulado ``Desarrollo de Modelo predictivo basado en machine learning para la identificación temprana de riesgo de morosidad en carteras crediticias'' que se ha presentado a la Universidad Tecnológica ECOTEC para obtener el título de Máster en Sistemas de Información con Mención en Inteligencia de Negocios.

Este trabajo es original y se ha desarrollado respetando los derechos intelectuales de terceros, conforme las citas que constan en el documento y cuyas fuentes se incorporan en las referencias bibliográficas.

En virtud de esta declaración, me responsabilizo del contenido, veracidad y alcance científico del trabajo de titulación referido.

\vspace{2cm}

\begin{flushright}
\_\_\_\_\_\_\_\_\_\_\_\_\_\_\_\_\_\_\_\_\_\_\_\_\_\_\_\\
Orlando Daniel Zambrano Zúñiga\\
C.I.: 0920470762\\
Fecha: \today
\end{flushright}

\newpage

% Dedicatoria
\chapter*{Dedicatoria}
\thispagestyle{empty}

\begin{flushright}
\emph{A mi familia, por su apoyo incondicional\\
durante todo este proceso académico.\\
Su paciencia y aliento han sido fundamentales\\
para la culminación de este proyecto.}
\end{flushright}

\newpage

% Agradecimientos
\chapter*{Agradecimientos}
\thispagestyle{empty}

Quiero expresar mi más sincero agradecimiento a todas las personas que hicieron posible la realización de este trabajo:

A mi tutor, Magíster Christian Merchán, por su guía, asesoramiento y crítica constructiva a lo largo de todo el proceso de investigación.

A la Universidad Tecnológica ECOTEC y a todos los docentes de la maestría, por compartir sus conocimientos y experiencias, contribuyendo significativamente a mi formación profesional.

A la empresa que me permitió acceder a sus datos, por la confianza depositada y la apertura para implementar el proyecto desarrollado.

A mis compañeros de maestría, por el enriquecedor intercambio de ideas y el apoyo mutuo durante estos años de estudio.

A todos quienes, de una u otra forma, contribuyeron al desarrollo exitoso de este trabajo.

\newpage

% Resumen
\chapter*{Resumen}
\thispagestyle{empty}

El presente trabajo aborda el desarrollo e implementación de un modelo predictivo basado en técnicas de machine learning para la identificación temprana del riesgo de morosidad en carteras crediticias del sector mayorista. Utilizando datos históricos del período 2017-2024 de una empresa distribuidora de productos de juguetería, hogar, aseo y cocina, se aplicó la metodología CRISP-DM para estructurar el proceso de minería de datos y desarrollo del modelo.

El estudio integra variables tradicionales de comportamiento de pago con factores contextuales como ubicación geográfica y estacionalidad, frecuentemente omitidos en modelos genéricos. Mediante la evaluación comparativa de múltiples algoritmos, se identificó XGBoost como la técnica con mejor rendimiento, alcanzando una precisión del 84\% en la predicción de riesgo de morosidad, superando significativamente los métodos tradicionales previamente utilizados.

La implementación del sistema incluyó el desarrollo de un dashboard interactivo y un sistema de alertas tempranas que amplió el tiempo promedio de anticipación de 3 a 15 días, permitiendo intervenciones preventivas efectivas que resultaron en una reducción del 18\% en la tasa general de morosidad durante el periodo piloto.

El análisis de costo-beneficio demuestra un retorno de inversión estimado de 6 meses, validando la viabilidad económica de la implementación de soluciones basadas en inteligencia de negocios en el contexto empresarial ecuatoriano.

\textbf{Palabras clave:} Machine learning, riesgo crediticio, morosidad, análisis predictivo, sector mayorista, cartera crediticia, CRISP-DM.

\newpage

% Abstract (en inglés)
\chapter*{Abstract}
\thispagestyle{empty}

This work addresses the development and implementation of a predictive model based on machine learning techniques for the early identification of default risk in credit portfolios in the wholesale sector. Using historical data from the 2017-2024 period from a distribution company of toys, household, cleaning, and kitchen products, the CRISP-DM methodology was applied to structure the data mining process and model development.

The study integrates traditional payment behavior variables with contextual factors such as geographic location and seasonality, frequently omitted in generic models. Through comparative evaluation of multiple algorithms, XGBoost was identified as the technique with the best performance, achieving 84\% accuracy in predicting default risk, significantly outperforming previously used traditional methods.

The system implementation included the development of an interactive dashboard and an early warning system that extended the average anticipation time from 3 to 15 days, allowing effective preventive interventions that resulted in an 18\% reduction in the overall delinquency rate during the pilot period.

The cost-benefit analysis demonstrates an estimated return on investment of 6 months, validating the economic viability of implementing business intelligence solutions in the Ecuadorian business context.

\textbf{Keywords:} Machine learning, credit risk, default, predictive analysis, wholesale sector, credit portfolio, CRISP-DM.

% ======================================
% ÍNDICES
% ======================================
\tableofcontents
\listoffigures
\listoftables

% Lista de códigos (si usas muchos ejemplos de SQL/Python)
\lstlistoflistings

\newpage

% ======================================
% CAPÍTULOS PRINCIPALES
% ======================================
\chapter{Introducción}

\section{Contexto y Planteamiento del Problema}

En la era de la transformación digital y el análisis de datos masivos, la gestión efectiva del riesgo crediticio representa una oportunidad ideal para la aplicación de técnicas avanzadas de inteligencia de negocios. El sector mayorista de productos de juguetería, hogar, aseo y cocina en Ecuador genera diariamente volúmenes significativos de datos transaccionales que, mediante técnicas adecuadas de análisis predictivo, pueden transformarse en insights valiosos para la toma de decisiones crediticias.

La empresa objeto de estudio, con más de 15 años de trayectoria en el mercado ecuatoriano, gestiona una cartera de 238 clientes activos distribuidos en 12 provincias del país, con un valor que supera el millón de dólares. A pesar de contar con un sistema de gestión de crédito tradicional, la compañía enfrenta desafíos significativos relacionados con la morosidad, que actualmente oscila entre el 12\% y el 18\% de la cartera total, dependiendo de la temporada.

Los métodos convencionales de evaluación crediticia utilizados por la empresa se basan principalmente en el análisis histórico retrospectivo y criterios subjetivos, lo que resulta en:

\begin{itemize}
    \item Baja capacidad para anticipar problemas de pago (promedio de 3 días de antelación)
    \item Precisión limitada en la identificación de clientes con alto riesgo (aproximadamente 60\%)
    \item Escasa consideración de factores contextuales como ubicación geográfica y estacionalidad
    \item Procesos manuales que consumen tiempo significativo del personal (promedio de 45 minutos por evaluación de cliente)
\end{itemize}

La disponibilidad de datos históricos desde 2017 proporciona un conjunto robusto de información para el entrenamiento de modelos de machine learning que puedan identificar patrones complejos en el comportamiento de pago. Estudios recientes demuestran que la integración de técnicas de inteligencia artificial en sistemas de información crediticia puede mejorar significativamente la precisión de las predicciones de riesgo comparado con métodos tradicionales \citep{kim2022credit, lessmann2015benchmarking}.

Este estudio aborda la necesidad de desarrollar un modelo predictivo adaptado específicamente a las particularidades del sector mayorista ecuatoriano, que permita identificar con anticipación el riesgo de morosidad y facilite la implementación de estrategias preventivas efectivas.

\section{Preguntas de Investigación}

\subsection{Pregunta Principal}

¿De qué manera la implementación de un sistema de información basado en técnicas de machine learning puede optimizar la predicción del riesgo de morosidad en una cartera crediticia del sector mayorista ecuatoriano?

\subsection{Preguntas Específicas}

\begin{enumerate}
    \item ¿Qué patrones y relaciones específicas en los datos históricos de comportamiento de pago pueden ser identificados mediante técnicas de minería de datos y cuál es su capacidad predictiva respecto a la morosidad futura?
    
    \item ¿Cómo pueden integrarse las variables estacionales y geográficas en un modelo de machine learning para mejorar la precisión de las predicciones de riesgo y en qué medida incrementan el poder predictivo del modelo?
    
    \item ¿Qué arquitectura de sistema de información es más efectiva para la implementación, mantenimiento y visualización de modelos predictivos en el contexto específico del comercio mayorista ecuatoriano?
    
    \item ¿Cuál es la efectividad del modelo predictivo implementado en comparación con los métodos tradicionales, medida en términos de precisión, tiempo de anticipación y reducción de la tasa de morosidad?
\end{enumerate}

\section{Idea a Defender}

La implementación de un modelo de información que integre técnicas avanzadas de machine learning y análisis predictivo permitirá identificar con una precisión superior al 80\% el riesgo de morosidad en la cartera de clientes mayoristas, facilitando la automatización y optimización de la gestión crediticia con un impacto económico positivo cuantificable.

\section{Objetivos}

\subsection{Objetivo General}

Desarrollar un modelo predictivo basado en técnicas de machine learning para la identificación temprana del riesgo de morosidad en la cartera crediticia de una empresa mayorista ecuatoriana, integrando datos históricos del período 2017-2024 para optimizar la toma de decisiones crediticias, con una precisión objetivo superior al 80\%.

\subsection{Objetivos Específicos}

\begin{enumerate}
    \item Implementar un ciclo de vida de datos que integre y normalice la información histórica de comportamiento de pago, considerando variables como cupo de crédito, ubicación geográfica y estacionalidad, para generar un conjunto de datos estructurado y de alta calidad para el análisis predictivo.
    
    \item Aplicar técnicas de minería de datos y análisis estadístico para identificar patrones significativos y variables predictivas en el comportamiento de la cartera, evaluando su correlación con la probabilidad de morosidad futura.
    
    \item Desarrollar e implementar modelos de machine learning que se integren al sistema de información existente para la predicción automatizada del riesgo de morosidad, evaluando comparativamente el rendimiento de diferentes algoritmos.
    
    \item Crear un dashboard interactivo que permita la visualización y monitoreo en tiempo real de los indicadores de riesgo crediticio generados por el modelo predictivo, facilitando la interpretación y utilización de resultados por parte de los usuarios.
    
    \item Validar la efectividad del modelo implementado mediante un análisis comparativo con los métodos tradicionales, cuantificando la mejora en precisión, tiempo de anticipación y reducción de morosidad durante un período piloto.
\end{enumerate}

\section{Justificación}

\subsection{Justificación Teórica}

Este proyecto contribuye al campo de la inteligencia de negocios mediante la validación de técnicas avanzadas de machine learning aplicadas específicamente al sector mayorista ecuatoriano. La investigación genera conocimiento sobre la efectividad de diferentes algoritmos predictivos en el contexto de carteras crediticias comerciales, aportando al cuerpo teórico de la ciencia de datos aplicada.

El estudio aborda un vacío significativo en la literatura existente sobre análisis crediticio en mercados latinoamericanos, donde la mayoría de los modelos se han desarrollado para contextos bancarios o de microfinanzas, con escasa atención a las particularidades del comercio mayorista. La integración de variables contextuales como estacionalidad y geografía en los modelos predictivos representa una contribución teórica relevante para el desarrollo de sistemas de información crediticia adaptados a realidades regionales específicas.

\subsection{Justificación Metodológica}

La investigación desarrolla una metodología sistemática para la elaboración de modelos predictivos en entornos con restricciones de datos, integrando técnicas de preparación de datos, selección de variables, entrenamiento de modelos y validación de resultados. Esta metodología podrá ser replicada y adaptada en contextos similares del sector comercial ecuatoriano y latinoamericano.

La aplicación estructurada de la metodología CRISP-DM, adaptada a las particularidades del sector mayorista, representa un aporte metodológico significativo para la implementación de proyectos de minería de datos en empresas comerciales de tamaño medio, constituyendo un marco de referencia para iniciativas similares.

\subsection{Justificación Práctica}

El sistema desarrollado permite a la empresa optimizar su gestión de riesgo crediticio mediante:

\begin{itemize}
    \item Reducción potencial de la tasa de morosidad, con un impacto económico directo en la recuperación de cartera
    
    \item Automatización de la evaluación de riesgo, reduciendo significativamente el tiempo operativo dedicado a esta tarea
    
    \item Mejora en la toma de decisiones crediticias, basada en predicciones objetivas y cuantificables
    
    \item Optimización de los procesos de cobranza, priorizando casos según su nivel de riesgo
    
    \item Implementación de estrategias preventivas personalizadas, basadas en perfiles de riesgo específicos
\end{itemize}

La solución desarrollada tiene un impacto económico directo cuantificable, estimado en una reducción de incobrables superior a \$50,000 anuales y una optimización operativa valorada en aproximadamente \$20,000 anuales en tiempo del personal.

\section{Alcance del Estudio}

El presente estudio tiene un alcance exploratorio-correlacional-explicativo, fundamentado en la metodología CRISP-DM (Cross-Industry Standard Process for Data Mining). Como señala \cite{hernandez2020metodologia}, este tipo de alcance permite no solo identificar patrones en los datos, sino también explicar las relaciones causales entre variables y desarrollar modelos con capacidad predictiva.

La investigación comprende el análisis de datos históricos de comportamiento crediticio de 238 clientes de una empresa mayorista ecuatoriana, durante el período 2017-2024, incluyendo aproximadamente 150,000 registros transaccionales. El estudio se enfoca en el desarrollo e implementación de un modelo predictivo para la identificación temprana del riesgo de morosidad, integrando variables tradicionales de comportamiento de pago con factores contextuales como ubicación geográfica y estacionalidad.

La validación del modelo se realiza mediante un período piloto de tres meses, durante el cual se evalúa su efectividad en términos de precisión, tiempo de anticipación y reducción de la tasa de morosidad, en comparación con los métodos tradicionales previamente utilizados por la empresa.

El alcance del proyecto incluye el desarrollo completo del ciclo de vida del modelo predictivo, desde la preparación y análisis de datos hasta la implementación de un sistema de información con dashboard interactivo y alertas tempranas, así como la documentación de resultados y recomendaciones para su mantenimiento y evolución futura.

\section{Consideraciones Éticas y de Confidencialidad}

Dado que este proyecto utiliza datos reales de una empresa comercial ecuatoriana, se han implementado rigurosos protocolos para garantizar la confidencialidad y el manejo ético de la información. Todos los datos presentados en este documento han sido anonimizados, eliminando cualquier información que pudiera identificar directamente a la empresa, sus clientes o empleados específicos.

El tratamiento de los datos personales y comerciales se realizó cumpliendo con los siguientes principios:

\begin{itemize}
    \item \textbf{Anonimización:} Todos los identificadores personales (nombres, cédulas, direcciones específicas) fueron reemplazados por códigos alfanuméricos o eliminados de los conjuntos de datos de análisis.
    
    \item \textbf{Propósito limitado:} Los datos fueron utilizados exclusivamente para los fines académicos y de investigación descritos en este proyecto, sin compartirlos con terceros ni utilizarlos para otros propósitos comerciales.
    
    \item \textbf{Minimización de datos:} Solo se procesaron los datos estrictamente necesarios para el desarrollo del modelo predictivo y la validación de resultados.
    
    \item \textbf{Seguridad de la información:} El almacenamiento y procesamiento de datos se realizó en infraestructura controlada con medidas de seguridad apropiadas, incluyendo cifrado y controles de acceso.
    
    \item \textbf{Consentimiento informado:} La empresa proporcionó autorización explícita para el uso de sus datos históricos con fines académicos, bajo acuerdo de confidencialidad.
\end{itemize}

Es importante destacar que el modelo predictivo desarrollado no utiliza ni considera variables protegidas que puedan generar discriminación (raza, género, religión, orientación política), enfocándose exclusivamente en indicadores objetivos de comportamiento crediticio y características comerciales relevantes para la evaluación de riesgo.

\section{Estructura del Documento}

El presente documento se estructura de la siguiente manera:

\begin{itemize}
    \item \textbf{Capítulo 1: Introducción}. Presenta el contexto, planteamiento del problema, objetivos, justificación y alcance del estudio.
    
    \item \textbf{Capítulo 2: Marco Teórico}. Desarrolla los fundamentos conceptuales de machine learning, análisis predictivo, gestión de riesgo crediticio y particularidades del sector mayorista ecuatoriano.
    
    \item \textbf{Capítulo 3: Metodología}. Detalla el diseño de la investigación y la aplicación de la metodología CRISP-DM en cada una de sus fases, incluyendo preparación de datos, modelado, evaluación e implementación.
    
    \item \textbf{Capítulo 4: Resultados y Discusión}. Presenta los resultados del análisis exploratorio, desarrollo del modelo predictivo, implementación del sistema y validación de efectividad, analizando su significado e implicaciones.
    
    \item \textbf{Capítulo 5: Conclusiones y Recomendaciones}. Sintetiza los principales hallazgos, conclusiones derivadas y recomendaciones para trabajo futuro.
\end{itemize}

Cada capítulo ha sido estructurado para abordar de manera sistemática los objetivos de investigación planteados, proporcionando una visión integral del desarrollo e implementación del modelo predictivo para la identificación temprana del riesgo de morosidad en carteras crediticias del sector mayorista ecuatoriano.

\chapter{Marco Teórico}

\section{Fundamentos de Machine Learning}
\subsection{Definición y conceptos básicos}
El machine learning o aprendizaje automático es una rama de la inteligencia artificial que se centra en el desarrollo de algoritmos capaces de aprender a partir de datos, identificar patrones y tomar decisiones con mínima intervención humana. Como señala \cite{mitchell1997machine}, "un programa de computadora aprende de la experiencia E con respecto a alguna clase de tareas T y medida de rendimiento P, si su rendimiento en tareas en T, medido por P, mejora con la experiencia E".

En el contexto de la gestión crediticia, el machine learning proporciona herramientas poderosas para analizar comportamientos históricos de pago y predecir patrones futuros con una precisión superior a los métodos estadísticos tradicionales \cite{garcia2024machine}.

\subsection{Tipos de aprendizaje}
\subsubsection{Aprendizaje supervisado}
El aprendizaje supervisado es paradigma donde el algoritmo aprende a partir de datos etiquetados, es decir, ejemplos que incluyen tanto las características de entrada como la salida deseada. En el contexto de predicción de morosidad, los modelos supervisados aprenden de registros históricos donde ya se conoce si un cliente incurrió en mora o no \cite{torres2023inteligencia}.

\subsubsection{Aprendizaje no supervisado}
A diferencia del enfoque supervisado, el aprendizaje no supervisado trabaja con datos no etiquetados, buscando descubrir estructuras, patrones o relaciones ocultas en los datos. La segmentación de clientes mediante técnicas de clustering es un ejemplo de aplicación de aprendizaje no supervisado en la gestión crediticia \cite{ramirez2023predictive}.

\subsubsection{Aprendizaje por refuerzo}
Este paradigma se basa en agentes que aprenden a tomar decisiones mediante la interacción con un entorno y la recepción de recompensas o penalizaciones. Aunque menos común en el análisis crediticio tradicional, tiene aplicaciones emergentes en estrategias dinámicas de gestión de cartera \cite{garcia2024machine}.

\section{Análisis Predictivo en Gestión Crediticia}
\subsection{Concepto de riesgo crediticio}
El riesgo crediticio se refiere a la posibilidad de pérdida derivada del incumplimiento de las obligaciones contractuales por parte de un deudor. En el contexto mayorista, este riesgo presenta características particulares relacionadas con los volúmenes de transacción, la estacionalidad y los factores geográficos que afectan a los comercios \cite{ramirez2023predictive}.

\subsection{Factores determinantes de la morosidad}
La morosidad en carteras crediticias comerciales está influenciada por múltiples factores que pueden clasificarse en:

\begin{itemize}
    \item \textbf{Factores internos del cliente:} Capacidad financiera, historial de pagos, antigüedad como cliente, volumen de compras.
    \item \textbf{Factores externos:} Condiciones económicas regionales, estacionalidad del negocio, competencia en el sector.
    \item \textbf{Factores transaccionales:} Frecuencia de compras, tamaño promedio de transacción, diversidad de productos adquiridos.
\end{itemize}

Según \cite{torres2023inteligencia}, la integración de estos factores en modelos predictivos puede mejorar la precisión de las predicciones hasta en un 40\% comparado con métodos tradicionales que consideran únicamente el historial de pagos.

\subsection{Indicadores clave en la evaluación crediticia}
Los principales indicadores utilizados tradicionalmente en la evaluación del riesgo crediticio incluyen:

\begin{itemize}
    \item Ratio de morosidad (cartera vencida/cartera total)
    \item Días promedio de atraso
    \item Frecuencia de incumplimientos
    \item Concentración de la cartera por cliente
    \item Rotación de inventario del cliente (como indicador de liquidez)
\end{itemize}

\section{Técnicas de Machine Learning aplicadas al riesgo crediticio}
\subsection{Algoritmos de clasificación}
\subsubsection{Regresión logística}
La regresión logística es uno de los métodos más utilizados en la evaluación de riesgo crediticio debido a su interpretabilidad y eficacia. Este algoritmo calcula la probabilidad de que un cliente pertenezca a una categoría específica (por ejemplo, moroso o no moroso) basándose en un conjunto de variables predictoras \cite{garcia2024machine}.

\subsubsection{Árboles de decisión y Random Forest}
Los árboles de decisión ofrecen una representación visual de las reglas de decisión que conducen a una clasificación. Random Forest, por su parte, construye múltiples árboles de decisión y combina sus resultados, lo que generalmente mejora la precisión y reduce el sobreajuste \cite{ramirez2023predictive}.

\subsubsection{Support Vector Machines (SVM)}
Las SVM buscan el hiperplano óptimo que separa las clases en el espacio de características. En contextos de riesgo crediticio, han demostrado alta precisión, especialmente cuando las relaciones entre variables no son lineales \cite{torres2023inteligencia}.

\subsubsection{Redes neuronales}
Las redes neuronales, especialmente las arquitecturas profundas, han ganado popularidad en el análisis crediticio debido a su capacidad para modelar relaciones complejas entre variables. Sin embargo, su interpretabilidad limitada puede ser un obstáculo en entornos donde la explicabilidad de las decisiones es crucial \cite{garcia2024machine}.

\subsection{Evaluación de modelos predictivos}
\subsubsection{Métricas de rendimiento}
La evaluación de modelos predictivos en el contexto de riesgo crediticio requiere considerar múltiples métricas:

\begin{itemize}
    \item \textbf{Precisión general:} Porcentaje de predicciones correctas.
    \item \textbf{Sensibilidad (recall):} Capacidad del modelo para identificar casos positivos (morosos).
    \item \textbf{Especificidad:} Capacidad del modelo para identificar casos negativos (no morosos).
    \item \textbf{Área bajo la curva ROC (AUC):} Medida integral del rendimiento del clasificador.
    \item \textbf{F1-Score:} Media armónica entre precisión y recall.
\end{itemize}

\subsubsection{Validación cruzada}
La validación cruzada es una técnica esencial para evaluar la capacidad de generalización de los modelos predictivos. Mediante la división de los datos en múltiples conjuntos de entrenamiento y prueba, permite estimar el rendimiento del modelo en datos no vistos previamente \cite{hernandez2020metodologia}.

\subsubsection{Ajuste de hiperparámetros}
El ajuste de hiperparámetros es crucial para optimizar el rendimiento de los modelos de machine learning. Técnicas como Grid Search y Random Search permiten explorar sistemáticamente diferentes configuraciones de parámetros para identificar las combinaciones que maximizan el rendimiento del modelo \cite{garcia2024machine}.

\subsection{XGBoost: Fundamentos y Aplicaciones}

\subsubsection{Fundamentos matemáticos de XGBoost}
XGBoost (eXtreme Gradient Boosting) representa una implementación optimizada del algoritmo de gradient boosting que ha demostrado un rendimiento excepcional en problemas de clasificación y regresión \cite{chen2016xgboost}. A diferencia de los algoritmos tradicionales de aprendizaje supervisado, XGBoost construye un conjunto de árboles de decisión de manera secuencial, donde cada árbol subsecuente intenta corregir los errores de los árboles anteriores.

La función objetivo que XGBoost minimiza durante el entrenamiento incorpora tanto el error de predicción como un término de regularización para controlar la complejidad del modelo y prevenir el sobreajuste \cite{chen2016xgboost}:

\begin{equation}
\text{Obj} = \sum_{i=1}^{n} L(y_i, \hat{y}_i) + \sum_{k=1}^{K} \Omega(f_k)
\end{equation}

Donde $L$ representa la función de pérdida que mide la discrepancia entre las predicciones y los valores reales, y $\Omega$ es el término de regularización que penaliza la complejidad de cada árbol. Esta formulación permite que el algoritmo encuentre un balance óptimo entre precisión predictiva y generalización.

El proceso de construcción de árboles en XGBoost utiliza una aproximación de segundo orden de la función objetivo mediante expansión de Taylor, lo que resulta en un algoritmo computacionalmente más eficiente que los métodos tradicionales de gradient boosting \cite{friedman2001greedy}. Esta aproximación permite calcular directamente la ganancia óptima de cada división del árbol, facilitando la poda de ramas que no contribuyen significativamente a la mejora del modelo.

XGBoost incorpora regularización L1 (Lasso) y L2 (Ridge) simultáneamente, lo que proporciona flexibilidad para controlar tanto el número de características activas como la magnitud de los pesos asignados a cada característica \cite{friedman2001greedy}. La regularización L1 promueve la esparsidad en las soluciones, eliminando efectivamente características irrelevantes, mientras que la regularización L2 suaviza los pesos para mejorar la estabilidad del modelo.

Entre las ventajas técnicas que distinguen a XGBoost de otros algoritmos de gradient boosting se encuentran: el manejo eficiente de valores faltantes mediante la asignación automática de una dirección predeterminada en cada división del árbol, la paralelización del proceso de construcción de árboles que aprovecha arquitecturas de múltiples núcleos, el sistema de caché optimizado que reduce el acceso a memoria y mejora la velocidad computacional, y el soporte nativo para validación cruzada durante el entrenamiento \cite{chen2016xgboost}.

En el contexto específico de la predicción de riesgo crediticio, XGBoost ha demostrado ventajas particulares debido a su capacidad para capturar interacciones complejas entre variables sin requerir especificación explícita de términos de interacción \cite{lessmann2015benchmarking}. Por ejemplo, puede identificar automáticamente que la combinación de días de atraso histórico con ubicación geográfica y estacionalidad tiene mayor poder predictivo que cada variable individual considerada aisladamente.

\subsubsection{Comparación con otros algoritmos en riesgo crediticio}
La selección del algoritmo apropiado para modelos de scoring crediticio requiere considerar múltiples dimensiones de evaluación, incluyendo precisión predictiva, interpretabilidad, velocidad computacional y robustez ante datos desbalanceados \cite{lessmann2015benchmarking}.

Los algoritmos de regresión logística, ampliamente utilizados en la industria financiera tradicional, ofrecen la ventaja de coeficientes fácilmente interpretables que pueden explicarse a reguladores y stakeholders \cite{baesens2003benchmarking}. Sin embargo, su capacidad predictiva está limitada por la asunción de relaciones lineales entre variables y la dificultad para capturar interacciones complejas sin ingeniería manual de características.

Los modelos de Random Forest, que construyen múltiples árboles de decisión sobre submuestras aleatorias de datos y características, proporcionan robustez contra el sobreajuste y capacidad para manejar relaciones no lineales \cite{breiman2001random}. No obstante, en estudios comparativos de riesgo crediticio, Random Forest ha mostrado un rendimiento ligeramente inferior a XGBoost, particularmente en datasets con alta dimensionalidad y desbalance de clases \cite{lessmann2015benchmarking}.

Las redes neuronales profundas han ganado atención reciente en aplicaciones de scoring crediticio debido a su capacidad para aprender representaciones jerárquicas de características \cite{xia2017novel}. Sin embargo, requieren conjuntos de datos significativamente más grandes para entrenamiento efectivo, presentan desafíos en términos de interpretabilidad, y su ventaja sobre métodos de ensemble como XGBoost no es consistente en datasets de tamaño mediano típicos del sector mayorista.

Las máquinas de vectores de soporte (SVM) con kernels no lineales pueden capturar fronteras de decisión complejas, pero su escalabilidad está limitada en datasets grandes y su proceso de entrenamiento es computacionalmente más costoso que XGBoost \cite{huang2007credit}. Además, la selección del kernel apropiado y el ajuste de hiperparámetros requiere considerable experimentación.

En estudios empíricos que comparan múltiples algoritmos en problemas de credit scoring, XGBoost ha demostrado consistentemente un rendimiento superior o comparable al mejor algoritmo en cada estudio, con la ventaja adicional de menor tiempo de entrenamiento y mayor flexibilidad en el manejo de tipos de datos mixtos \cite{lessmann2015benchmarking,xia2017novel}.

\section{Metodología CRISP-DM}
\subsection{Descripción y fases}
CRISP-DM (Cross-Industry Standard Process for Data Mining) es una metodología ampliamente adoptada para proyectos de minería de datos y machine learning. Consta de seis fases interconectadas \cite{hernandez2020metodologia}:

\begin{enumerate}
    \item \textbf{Comprensión del negocio:} Definición de objetivos y requerimientos desde una perspectiva empresarial.
    \item \textbf{Comprensión de los datos:} Recolección inicial de datos y familiarización con su estructura y calidad.
    \item \textbf{Preparación de los datos:} Limpieza, transformación y selección de variables.
    \item \textbf{Modelado:} Selección y aplicación de técnicas de modelado.
    \item \textbf{Evaluación:} Revisión del modelo y verificación del cumplimiento de los objetivos de negocio.
    \item \textbf{Despliegue:} Implementación del modelo en el entorno de producción.
\end{enumerate}

\subsection{Justificación de la elección metodológica}
CRISP-DM ha sido seleccionada como metodología para este proyecto debido a su enfoque estructurado y orientado a resultados empresariales. A diferencia de otras metodologías como KDD (Knowledge Discovery in Databases) o SEMMA (Sample, Explore, Modify, Model, Assess), CRISP-DM pone un mayor énfasis en la comprensión del contexto de negocio antes de iniciar el análisis técnico \cite{hernandez2020metodologia}.

Además, CRISP-DM facilita un enfoque iterativo que permite refinar continuamente el modelo basándose en los resultados obtenidos, lo que resulta especialmente valioso en el contexto dinámico del riesgo crediticio en el sector mayorista ecuatoriano.

\subsection{Casos de éxito y validación empírica}
La metodología CRISP-DM ha sido aplicada exitosamente en numerosos proyectos de minería de datos en contextos empresariales diversos, validando su utilidad como marco estructurado para iniciativas de analítica avanzada \cite{martinez2019crisp}.

En el sector financiero, un caso documentado en instituciones bancarias europeas describe la implementación de CRISP-DM para desarrollar modelos de detección de fraude en transacciones con tarjeta de crédito \cite{wirth2000crisp}. El proyecto siguió las seis fases metodológicas, comenzando con la comprensión del negocio donde se identificó que el costo de falsos positivos (bloqueo de transacciones legítimas) era significativamente mayor que el costo de falsos negativos. Esta comprensión del contexto de negocio influyó directamente en la selección de métricas de evaluación y umbrales de decisión en fases posteriores.

En el sector retail, estudios de caso en cadenas de supermercados latinoamericanos han documentado la aplicación de CRISP-DM para desarrollar sistemas de recomendación personalizados y modelos de predicción de demanda \cite{chapman2000crisp}. La fase de comprensión de los datos reveló patrones de estacionalidad específicos del mercado local que no estaban presentes en modelos desarrollados en otras geografías, lo que llevó a la incorporación de features temporales personalizadas en la fase de preparación de datos.

En aplicaciones de credit scoring específicamente, casos en instituciones de microfinanzas asiáticas ilustran la aplicación sistemática de CRISP-DM para desarrollar modelos predictivos de morosidad en poblaciones con escaso historial crediticio formal \cite{martinez2019crisp}. La metodología facilitó la iteración entre las fases de modelado y evaluación, donde múltiples algoritmos fueron probados y comparados antes de seleccionar el modelo final para despliegue.

Un aspecto crítico del éxito documentado en estos casos es la naturaleza iterativa de CRISP-DM, que permite regresar a fases anteriores cuando se descubren insights nuevos o limitaciones en fases posteriores \cite{wirth2000crisp}. Por ejemplo, durante la fase de modelado podría descubrirse que ciertas variables importantes no fueron adecuadamente transformadas en la fase de preparación de datos, permitiendo iterar hacia atrás para mejorar el procesamiento.

\section{Sistemas de Información para la gestión de riesgo crediticio}
\subsection{Arquitecturas de sistemas de información}
Las arquitecturas modernas para sistemas de gestión de riesgo crediticio generalmente siguen un modelo de capas que incluye:

\begin{itemize}
    \item \textbf{Capa de datos:} Almacenamiento y gestión de datos transaccionales e históricos.
    \item \textbf{Capa de procesamiento:} Implementación de algoritmos de análisis y predicción.
    \item \textbf{Capa de presentación:} Interfaces para la visualización de resultados y toma de decisiones.
\end{itemize}

En el contexto específico del sector mayorista, estas arquitecturas deben adaptarse para manejar volúmenes variables de transacciones y considerar factores como la estacionalidad y la distribución geográfica \cite{ramirez2023predictive}.

\subsection{Integración de modelos predictivos en sistemas existentes}
La integración exitosa de modelos predictivos en sistemas de información existentes requiere considerar aspectos como:

\begin{itemize}
    \item \textbf{Interoperabilidad:} Capacidad del modelo para interactuar con los sistemas de gestión transaccional existentes.
    \item \textbf{Escalabilidad:} Adaptación a volúmenes crecientes de datos y usuarios.
    \item \textbf{Mantenibilidad:} Facilidad para actualizar y refinar el modelo con nuevos datos.
    \item \textbf{Seguridad:} Protección de datos sensibles de clientes e información financiera.
\end{itemize}

\subsection{Visualización de datos y dashboards analíticos}
Los dashboards analíticos constituyen una interfaz crucial entre los modelos predictivos y los usuarios finales. Según \cite{torres2023inteligencia}, las características esenciales de un dashboard efectivo para la gestión de riesgo crediticio incluyen:

\begin{itemize}
    \item Representación visual intuitiva de indicadores de riesgo
    \item Capacidad de filtrado por variables críticas (geografía, segmento, etc.)
    \item Alertas automáticas basadas en umbrales predefinidos
    \item Visualización de tendencias temporales y patrones estacionales
    \item Acceso a diferentes niveles de detalle (drill-down)
\end{itemize}

\section{El sector mayorista en Ecuador: contexto y particularidades}
\subsection{Caracterización del sector mayorista de productos de juguetería, hogar, aseo y cocina}
El sector mayorista de productos de juguetería, hogar, aseo y cocina en Ecuador presenta características específicas que influyen en la dinámica crediticia:

\begin{itemize}
    \item Alta estacionalidad, con picos de demanda en periodos festivos
    \item Sensibilidad a fluctuaciones económicas regionales
    \item Cadenas de suministro complejas con componentes de importación
    \item Márgenes variables según categoría de producto
\end{itemize}

\subsection{Dinámica crediticia en el sector mayorista ecuatoriano}
Las prácticas crediticias en el sector mayorista ecuatoriano se caracterizan por:

\begin{itemize}
    \item Plazos de pago que típicamente oscilan entre 30 y 90 días
    \item Variaciones regionales significativas en comportamiento de pago
    \item Importancia de las relaciones personales en la evaluación crediticia informal
    \item Alta competencia que presiona a los proveedores a ofrecer condiciones crediticias favorables
\end{itemize}

Estas particularidades deben ser consideradas en el diseño de modelos predictivos adaptados a la realidad local, ya que difieren de patrones globales generalizados \cite{ramirez2023predictive}.

\subsection{Desafíos específicos en la gestión de riesgo crediticio mayorista}
La gestión de riesgo crediticio en el sector mayorista ecuatoriano enfrenta desafíos particulares:

\begin{itemize}
    \item Limitada formalización financiera de muchos comercios minoristas
    \item Escasa disponibilidad de información crediticia centralizada
    \item Variabilidad regional en prácticas comerciales y cumplimiento
    \item Impacto de eventos económicos locales en la capacidad de pago
\end{itemize}

Estos desafíos representan tanto obstáculos como oportunidades para la implementación de modelos predictivos avanzados que puedan capturar y adaptarse a estas particularidades \cite{torres2023inteligencia}.

\subsection{Aplicaciones de Machine Learning en gestión crediticia latinoamericana}
La adopción de técnicas de machine learning para la gestión de riesgo crediticio en Latinoamérica ha experimentado un crecimiento significativo en la última década, impulsada por la digitalización del sector financiero y la disponibilidad creciente de datos transaccionales \cite{barroso2022machine}.

En Brasil, diversos estudios han documentado la implementación exitosa de modelos predictivos en instituciones de microfinanzas, donde los métodos tradicionales de scoring crediticio mostraban limitaciones debido a la escasez de historial crediticio formal de los clientes \cite{oreski2014genetic}. La incorporación de variables alternativas como datos de transacciones móviles, patrones de consumo de servicios públicos y comportamiento de compra permitió mejorar significativamente la precisión de las predicciones de morosidad.

En Colombia, investigaciones recientes han explorado la aplicación de técnicas de ensemble learning, incluyendo XGBoost y Random Forest, en carteras comerciales de pequeñas y medianas empresas \cite{pena2021credit}. Los resultados indicaron que la incorporación de variables macroeconómicas regionales y factores de estacionalidad sectorial mejoró la capacidad predictiva de los modelos en un rango del quince al veinte por ciento en comparación con modelos tradicionales basados únicamente en variables financieras.

En el contexto ecuatoriano específicamente, el sistema financiero ha comenzado a explorar metodologías de analítica avanzada, aunque la adopción en el sector mayorista B2B permanece en etapas iniciales. La particularidad del mercado ecuatoriano, caracterizado por una economía dolarizada y alta informalidad en ciertos sectores, presenta desafíos únicos que requieren adaptaciones de los modelos desarrollados en otras geografías.

Un factor diferenciador en los modelos de riesgo crediticio para Latinoamérica es la necesidad de incorporar variables contextuales que capturan la volatilidad económica característica de mercados emergentes \cite{barroso2022machine}. La inclusión de indicadores como fluctuaciones en tasas de cambio paralelas, variaciones en precios de commodities relevantes al sector y ciclos políticos ha demostrado mejorar la robustez de las predicciones durante períodos de inestabilidad.

La literatura también destaca la importancia de considerar factores culturales en el comportamiento de pago en Latinoamérica, donde las relaciones comerciales personalizadas y la negociación flexible de términos de pago son más prevalentes que en mercados desarrollados \cite{pena2021credit}. Los modelos que incorporan variables que capturan la antigüedad y calidad de la relación comercial han mostrado mejor desempeño que aquellos que se enfocan exclusivamente en métricas financieras.
\chapter{Metodología}

\section{Diseño de la Investigación}
\subsection{Tipo y alcance de la investigación}
La presente investigación se enmarca en un alcance exploratorio-correlacional-explicativo. Como señala \cite{hernandez2020metodologia}, este enfoque multicapa permite:

\begin{itemize}
    \item \textbf{Fase exploratoria:} Identificar variables y patrones iniciales en los datos históricos de comportamiento crediticio, examinando factores que podrían influir en la morosidad.
    \item \textbf{Fase correlacional:} Establecer relaciones estadísticas entre las variables identificadas y el comportamiento de pago, cuantificando su influencia relativa.
    \item \textbf{Fase explicativa:} Desarrollar modelos predictivos que expliquen y anticipen el riesgo de morosidad, proporcionando una base para intervenciones preventivas.
\end{itemize}

Este enfoque resulta particularmente adecuado para abordar la complejidad del comportamiento crediticio en el sector mayorista, donde intervienen múltiples variables interrelacionadas.

\subsection{Enfoque metodológico}
La investigación adopta un enfoque predominantemente cuantitativo, basado en el análisis de datos históricos de comportamiento de pago y variables asociadas. Este enfoque se complementa con elementos cualitativos para la interpretación contextual de los resultados, considerando factores del sector mayorista ecuatoriano que podrían no estar completamente capturados en los datos numéricos.

\section{Aplicación de la Metodología CRISP-DM}
\subsection{Fase 1: Comprensión del negocio}
\subsubsection{Contexto organizacional}
La empresa objeto de estudio opera en el sector mayorista de productos de juguetería, hogar, aseo y cocina en Ecuador. Con una trayectoria de más de 15 años en el mercado, gestiona una cartera crediticia que supera el millón de dólares, distribuida entre 238 clientes activos a nivel nacional.

\subsubsection{Objetivos de negocio}
Los objetivos de negocio que motivan esta investigación incluyen:

\begin{itemize}
    \item Reducir la tasa de morosidad en al menos un 20\% respecto a los niveles actuales
    \item Optimizar la asignación de cupos de crédito basada en análisis predictivo del comportamiento
    \item Implementar un sistema de alertas tempranas que permita intervenciones preventivas
    \item Automatizar el proceso de evaluación de riesgo para nuevos clientes y ampliaciones de cupo
\end{itemize}

\subsubsection{Evaluación de la situación actual}
Actualmente, la empresa utiliza métodos tradicionales para la evaluación crediticia, basados principalmente en:

\begin{itemize}
    \item Historial de pagos
    \item Referencias comerciales
    \item Tiempo como cliente
    \item Volumen promedio de compras
\end{itemize}

Este enfoque presenta limitaciones significativas:

\begin{itemize}
    \item Baja capacidad predictiva (precisión estimada del 60\%)
    \item Evaluación reactiva más que preventiva
    \item Escasa consideración de factores externos (estacionalidad, geografía)
    \item Proceso manual con alta dependencia del criterio individual
\end{itemize}

\subsubsection{Determinación de objetivos de minería de datos}
Con base en la comprensión del negocio, se establecen los siguientes objetivos específicos para el proceso de minería de datos:

\begin{itemize}
    \item Identificar los factores con mayor poder predictivo para anticipar riesgo de morosidad
    \item Desarrollar un modelo predictivo con precisión superior al 80\%
    \item Segmentar la cartera de clientes según perfiles de riesgo
    \item Generar indicadores tempranos de alerta para cada segmento
\end{itemize}

\subsection{Fase 2: Comprensión de los datos}
\subsubsection{Descripción de las fuentes de datos}
Los datos para esta investigación provienen de múltiples fuentes internas de la empresa:

\begin{itemize}
    \item \textbf{Sistema ERP:} Datos transaccionales de ventas, devoluciones y pagos desde 2017
    \item \textbf{CRM:} Información de clientes, incluyendo datos demográficos y geográficos
    \item \textbf{Sistema de cobranzas:} Registros históricos de gestión de cobro y comportamiento de pago
    \item \textbf{Hojas de cálculo complementarias:} Registros manuales de seguimiento de cartera
\end{itemize}

\subsubsection{Exploración inicial de datos}
La exploración inicial revela un conjunto de datos con las siguientes características:

\begin{itemize}
    \item \textbf{Volumen:} Aproximadamente 150,000 registros transaccionales
    \item \textbf{Período:} Datos históricos desde enero 2017 hasta diciembre 2024
    \item \textbf{Granularidad:} Nivel transaccional (facturas, pagos, notas de crédito)
    \item \textbf{Dimensiones geográficas:} Distribución en 12 provincias del Ecuador
    \item \textbf{Categorías de productos:} 4 líneas principales (juguetería, hogar, aseo, cocina)
\end{itemize}

\subsubsection{Verificación de calidad de datos}
La evaluación preliminar de calidad de datos identifica los siguientes aspectos:

\begin{table}[ht]
\centering
\begin{tabular}{|p{4cm}|p{3cm}|p{7cm}|}
\hline
\textbf{Aspecto} & \textbf{Estado} & \textbf{Observaciones} \\
\hline
Completitud & Parcial & Datos faltantes en aproximadamente 12\% de los registros, principalmente en campos descriptivos y categorización de productos \\
\hline
Consistencia & Moderada & Inconsistencias en la codificación de regiones geográficas y en la clasificación de motivos de retraso \\
\hline
Precisión & Alta en datos financieros & Los montos y fechas de transacciones muestran alta precisión, mientras que los datos cualitativos presentan mayor variabilidad \\
\hline
Actualización & Diaria para transacciones & Los sistemas transaccionales se actualizan diariamente, mientras que la información complementaria tiene actualizaciones variables \\
\hline
\end{tabular}
\caption{Evaluación de calidad de datos}
\end{table}

\subsection{Fase 3: Preparación de los datos}
\subsubsection{Selección de datos}
Con base en la exploración inicial y los objetivos establecidos, se seleccionan las siguientes variables para el análisis:

\begin{table}[ht]
\centering
\begin{tabular}{|p{4cm}|p{3cm}|p{7cm}|}
\hline
\textbf{Variable} & \textbf{Tipo} & \textbf{Descripción} \\
\hline
ID\_Cliente & Categórica & Identificador único de cliente \\
\hline
Región & Categórica & Ubicación geográfica (provincia) \\
\hline
Antigüedad & Numérica & Tiempo como cliente (meses) \\
\hline
Promedio\_Compra & Numérica & Monto promedio de compra mensual \\
\hline
Frecuencia\_Compra & Numérica & Número promedio de compras por mes \\
\hline
Días\_Promedio\_Pago & Numérica & Tiempo promedio para completar pagos \\
\hline
Máximo\_Días\_Atraso & Numérica & Máximo retraso histórico en días \\
\hline
Porcentaje\_Devoluciones & Numérica & Porcentaje de mercadería devuelta \\
\hline
Categoría\_Principal & Categórica & Línea de producto predominante \\
\hline
Índice\_Estacionalidad & Numérica & Variación estacional en compras \\
\hline
Historial\_Mora & Categórica & Clasificación histórica de comportamiento \\
\hline
\end{tabular}
\caption{Variables seleccionadas para el análisis}
\end{table}

\subsubsection{Limpieza de datos}
El proceso de limpieza incluye las siguientes acciones:

\begin{itemize}
    \item \textbf{Tratamiento de valores faltantes:} Imputación mediante técnicas estadísticas para variables numéricas y moda para variables categóricas
    \item \textbf{Identificación y manejo de valores atípicos:} Aplicación del método IQR (Rango Intercuartílico) para detectar outliers y su normalización
    \item \textbf{Estandarización de codificaciones:} Unificación de categorías geográficas y de productos
    \item \textbf{Validación de coherencia temporal:} Corrección de inconsistencias en secuencias temporales de transacciones
\end{itemize}

\subsubsection{Construcción y transformación de variables}
A partir de los datos disponibles, se derivan las siguientes variables adicionales:

\begin{itemize}
    \item \textbf{Ratio\_Pago\_Plazo:} Relación entre días reales de pago y plazo acordado
    \item \textbf{Variabilidad\_Pago:} Desviación estándar en comportamiento de pago
    \item \textbf{Índice\_Concentración\_Producto:} Diversificación de compras entre categorías
    \item \textbf{Tendencia\_Compra:} Pendiente de evolución de montos de compra (últimos 6 meses)
    \item \textbf{Índice\_Cumplimiento:} Ratio ponderado de cumplimiento de compromisos de pago
\end{itemize}

Las transformaciones aplicadas incluyen:

\begin{itemize}
    \item Normalización Min-Max para variables numéricas
    \item Codificación One-Hot para variables categóricas
    \item Transformación logarítmica para variables con distribución sesgada
    \item Discretización de variables continuas para análisis específicos
\end{itemize}

\subsubsection{Integración de datos}
La integración de las diversas fuentes se realiza mediante un proceso ETL (Extracción, Transformación, Carga) que combina:

\begin{itemize}
    \item Exportación programada de datos transaccionales
    \item Carga incremental de nuevos registros
    \item Consolidación en una estructura unificada de datos
    \item Verificación de integridad referencial entre fuentes
\end{itemize}

El resultado es un conjunto de datos integrado con 238 registros (uno por cliente) y 32 variables (originales y derivadas).

\subsection{Fase 4: Modelado}
\subsubsection{Selección de técnicas de modelado}
Con base en los objetivos del proyecto y la naturaleza de los datos, se seleccionan las siguientes técnicas de modelado:

\begin{table}[ht]
\centering
\begin{tabular}{|p{3.5cm}|p{3.5cm}|p{7cm}|}
\hline
\textbf{Técnica} & \textbf{Propósito} & \textbf{Justificación} \\
\hline
Regresión Logística & Modelo base de clasificación & Ofrece interpretabilidad y establece una línea base de rendimiento \\
\hline
Random Forest & Clasificación avanzada & Alta precisión y capacidad para manejar variables correlacionadas \\
\hline
XGBoost & Clasificación optimizada & Rendimiento superior en problemas de clasificación binaria \\
\hline
K-Means & Segmentación de clientes & Identificación de grupos naturales en la cartera según perfil de riesgo \\
\hline
Redes Neuronales & Exploración de relaciones complejas & Capacidad para capturar interacciones no lineales entre variables \\
\hline
\end{tabular}
\caption{Técnicas de modelado seleccionadas}
\end{table}

\subsubsection{Diseño de pruebas}
El enfoque de validación incluye:

\begin{itemize}
    \item \textbf{División de datos:} 70\% entrenamiento, 30\% prueba
    \item \textbf{Validación cruzada:} K-fold (k=5) para estimación robusta del rendimiento
    \item \textbf{Técnicas de remuestreo:} SMOTE para abordar el desbalance de clases
    \item \textbf{Evaluación secuencial:} Entrenamiento progresivo con datos cronológicos para simular implementación real
\end{itemize}

\subsubsection{Construcción de modelos}
Para cada técnica seleccionada, se implementan los siguientes pasos:

\begin{enumerate}
    \item Inicialización con configuración base
    \item Entrenamiento inicial con conjunto de entrenamiento
    \item Ajuste de hiperparámetros mediante Grid Search o Bayesian Optimization
    \item Reentrenamiento con configuración optimizada
    \item Evaluación con métricas múltiples
\end{enumerate}

\subsubsection{Evaluación y comparación de modelos}
La evaluación de los modelos se realiza utilizando las siguientes métricas:

\begin{itemize}
    \item Exactitud (Accuracy)
    \item Precisión (Precision)
    \item Sensibilidad (Recall)
    \item F1-Score
    \item Área bajo la curva ROC (AUC)
    \item Pérdida logarítmica (Log Loss)
    \item Tiempo de entrenamiento y predicción
\end{itemize}

Adicionalmente, se evalúa la relevancia de las variables (feature importance) para cada modelo, lo que proporciona insights sobre los factores más determinantes en la predicción de morosidad.

\subsection{Fase 5: Evaluación}
\subsubsection{Evaluación de resultados respecto a objetivos de negocio}
Los resultados obtenidos se evalúan en relación con los objetivos de negocio establecidos:

\begin{itemize}
    \item \textbf{Reducción de morosidad:} Estimación de impacto potencial basada en la capacidad predictiva
    \item \textbf{Optimización de asignación de cupo:} Evaluación de recomendaciones generadas por el modelo
    \item \textbf{Efectividad de alertas tempranas:} Tiempo de anticipación logrado por las predicciones
    \item \textbf{Automatización del proceso:} Viabilidad técnica y operativa de la implementación
\end{itemize}

\subsubsection{Revisión del proceso}
Se realiza una revisión integral del proceso seguido, identificando:

\begin{itemize}
    \item Lecciones aprendidas en cada fase
    \item Desafíos metodológicos enfrentados
    \item Adaptaciones realizadas a la metodología estándar
    \item Áreas de mejora para futuras iteraciones
\end{itemize}

\subsection{Fase 6: Implementación}
\subsubsection{Plan de implementación}
La implementación del modelo seleccionado contempla las siguientes etapas:

\begin{enumerate}
    \item \textbf{Desarrollo de API:} Creación de interfaces programáticas para integración con sistemas existentes
    \item \textbf{Diseño de dashboard:} Desarrollo de interfaz de visualización para usuarios finales
    \item \textbf{Automatización de actualización:} Implementación de procesos de reentrenamiento periódico
    \item \textbf{Documentación técnica:} Elaboración de manuales y guías de referencia
    \item \textbf{Capacitación:} Formación a usuarios en interpretación y uso de resultados
\end{enumerate}

\subsubsection{Monitoreo y mantenimiento}
El plan de monitoreo y mantenimiento incluye:

\begin{itemize}
    \item Seguimiento continuo del rendimiento del modelo
    \item Alertas automáticas ante degradación de precisión
    \item Actualización trimestral con nuevos datos de comportamiento
    \item Revisión semestral de variables y parámetros
    \item Procedimientos de backup y recuperación
\end{itemize}

\section{Herramientas y Tecnologías}
\subsection{Software utilizado}
Para la implementación del proyecto se utilizan las siguientes herramientas:

\begin{table}[ht]
\centering
\begin{tabular}{|p{3cm}|p{3cm}|p{8cm}|}
\hline
\textbf{Herramienta} & \textbf{Versión} & \textbf{Propósito} \\
\hline
Python & 3.9 & Lenguaje principal para análisis de datos y modelado \\
\hline
Pandas & 1.4.2 & Manipulación y análisis de datos estructurados \\
\hline
Scikit-learn & 1.0.2 & Implementación de algoritmos de machine learning \\
\hline
XGBoost & 1.5.1 & Modelo avanzado de boosting para clasificación \\
\hline
TensorFlow & 2.9.0 & Implementación de redes neuronales \\
\hline
SQL Server & 2022 & Almacenamiento de datos integrados \\
\hline
Power BI & 2024 & Desarrollo de dashboards y visualizaciones \\
\hline
Git & 2.35.1 & Control de versiones de código y documentación \\
\hline
\end{tabular}
\caption{Software utilizado en el proyecto}
\end{table}

\subsection{Infraestructura tecnológica}
La arquitectura tecnológica implementada comprende:

\begin{itemize}
    \item \textbf{Servidores:} Entorno virtualizado con sistema operativo Linux
    \item \textbf{Almacenamiento:} Sistema de bases de datos relacional con respaldos incrementales
    \item \textbf{Procesamiento:} Implementación optimizada para ejecución eficiente de algoritmos
    \item \textbf{Comunicación:} Interfaces API REST para integración con sistemas empresariales
    \item \textbf{Seguridad:} Encriptación de datos sensibles y control de acceso basado en roles
\end{itemize}

\section{Consideraciones Éticas y de Seguridad}
\subsection{Manejo de datos sensibles}
El proyecto implementa las siguientes medidas para el manejo ético de datos:

\begin{itemize}
    \item Anonimización de identificadores personales directos
    \item Agregación de datos para análisis que no requieren granularidad individual
    \item Almacenamiento encriptado de información sensible
    \item Política de acceso restrictivo basada en necesidad de conocimiento
    \item Documentación de flujos de datos y responsables de procesamiento
\end{itemize}

\subsection{Sesgos potenciales y estrategias de mitigación}
Se identifican los siguientes sesgos potenciales y sus correspondientes estrategias de mitigación:

\begin{table}[ht]
\centering
\begin{tabular}{|p{4cm}|p{5cm}|p{5cm}|}
\hline
\textbf{Tipo de sesgo} & \textbf{Manifestación potencial} & \textbf{Estrategia de mitigación} \\
\hline
Sesgo geográfico & Predicciones más precisas para regiones con mayor representación en los datos & Estratificación y ponderación de muestras por región \\
\hline
Sesgo temporal & Menor precisión en periodos atípicos o estacionales & Inclusión explícita de variables temporales y estacionales \\
\hline
Sesgo de selección & Sobrerrepresentación de clientes con mayor historial & Técnicas de balanceo para nuevos clientes vs. antiguos \\
\hline
Sesgo de confirmación & Tendencia a mantener patrones de evaluación preexistentes & Validación cruzada y evaluación por múltiples criterios \\
\hline
\end{tabular}
\caption{Sesgos potenciales y estrategias de mitigación}
\end{table}

\subsection{Cumplimiento normativo}
El desarrollo e implementación del modelo predictivo considera el cumplimiento de:

\begin{itemize}
    \item Ley Orgánica de Protección de Datos Personales del Ecuador
    \item Normativas de la Superintendencia de Compañías sobre reportes de cartera
    \item Políticas internas de confidencialidad y manejo de información
    \item Estándares internacionales de seguridad de información (ISO 27001)
\end{itemize}
\chapter{Resultados y Discusión}
\section{Implementación del Data Warehouse}
\subsection{Métricas de carga y volumetría}
La implementación completa del Data Warehouse procesó exitosamente los datos históricos del período 2017-2024 de la empresa distribuidora. El proceso de carga inicial tomó aproximadamente 85 minutos, generando una base de datos analítica con las siguientes características de volumetría:

\begin{table}[ht]
\centering
\begin{tabular}{|l|r|r|}
\hline
\textbf{Componente} & \textbf{Registros} & \textbf{Tamaño (MB)} \\
\hline
\multicolumn{3}{|c|}{\textit{Dimensiones}} \\
\hline
DimTiempo & 5,844 & 0.8 \\
DimCliente & 1,407 & 2.1 \\
DimProducto & 4,127 & 3.5 \\
DimPuntoVenta & 2 & < 0.1 \\
DimCampana & 83 & 0.1 \\
DimTipoPago & 8 & < 0.1 \\
DimEstado & 23 & < 0.1 \\
\hline
\textbf{Subtotal Dimensiones} & \textbf{11,494} & \textbf{6.6} \\
\hline
\multicolumn{3}{|c|}{\textit{Tablas de Hechos}} \\
\hline
FactFacturas & 85,242 & 348.5 \\
FactPagos & 14,356 & 78.2 \\
FactDevoluciones & 0 & 0.0 \\
\hline
\textbf{Subtotal Hechos} & \textbf{99,598} & \textbf{426.7} \\
\hline
\multicolumn{3}{|c|}{\textit{Machine Learning}} \\
\hline
DatasetMorosidad & 568 & 0.4 \\
Predicciones & 1,892 & 1.2 \\
ModelosRegistro & 15 & < 0.1 \\
\hline
\textbf{Subtotal ML} & \textbf{2,475} & \textbf{1.6} \\
\hline
\multicolumn{3}{|c|}{\textit{Control y Auditoría}} \\
\hline
LogEjecucion & 147 & 0.2 \\
CalidadDatos & 89 & 0.1 \\
ControlCargaIncremental & 8 & < 0.1 \\
ConfiguracionProcesos & 6 & < 0.1 \\
ErroresDetallados & 12 & < 0.1 \\
\hline
\textbf{Subtotal Control} & \textbf{262} & \textbf{0.4} \\
\hline
\hline
\textbf{TOTAL GENERAL} & \textbf{113,829} & \textbf{435.3} \\
\hline
\end{tabular}
\caption{Volumetría del Data Warehouse implementado}
\label{tab:volumetria_dw}
\end{table}

La tabla \ref{tab:volumetria_dw} muestra que el grueso del almacenamiento corresponde a las tablas de hechos (98 por ciento del espacio total), lo cual es esperado en una arquitectura dimensional dado que estas tablas almacenan datos a nivel transaccional granular. Las dimensiones, por su naturaleza consolidada, representan apenas el 1.5 por ciento del espacio total.

\subsection{Calidad de datos y métricas de validación}

El sistema de control de calidad implementado en el esquema ETL permitió monitorear continuamente la integridad y consistencia de los datos cargados. Se establecieron 23 reglas de validación automáticas organizadas en cuatro categorías:

\begin{enumerate}
    \item \textbf{Completitud:} Verificación de campos obligatorios no nulos
    \item \textbf{Consistencia:} Validación de relaciones entre tablas y rangos de valores
    \item \textbf{Precisión:} Comprobación de formatos y tipos de datos
    \item \textbf{Integridad referencial:} Verificación de existencia de claves foráneas
\end{enumerate}

\begin{table}[ht]
\centering
\begin{tabular}{|l|r|r|r|}
\hline
\textbf{Categoría} & \textbf{Registros} & \textbf{Errores} & \textbf{Calidad (\%)} \\
\hline
Completitud campos clave & 99,598 & 0 & 100.0 \\
Completitud campos descriptivos & 99,598 & 1,247 & 98.7 \\
Consistencia de fechas & 99,598 & 83 & 99.9 \\
Consistencia de montos & 99,598 & 0 & 100.0 \\
Integridad referencial Cliente & 99,598 & 14 & 99.99 \\
Integridad referencial Producto & 85,242 & 127 & 99.85 \\
Rangos válidos de mora & 85,242 & 0 & 100.0 \\
Formato de identificaciones & 1,407 & 23 & 98.4 \\
\hline
\textbf{Promedio General} & - & - & \textbf{99.6} \\
\hline
\end{tabular}
\caption{Métricas de calidad de datos del Data Warehouse}
\end{table}

El nivel de calidad promedio de 99.6 por ciento indica que el proceso ETL logró mantener la integridad de los datos durante la transformación desde los sistemas transaccionales. Los errores detectados corresponden principalmente a datos faltantes en campos descriptivos secundarios que no afectan el análisis de morosidad.

\subsection{Rendimiento del proceso ETL}

Se realizaron pruebas de rendimiento midiendo los tiempos de ejecución de cada fase del proceso ETL en diferentes volúmenes de datos:

\begin{table}[ht]
\centering
\begin{tabular}{|l|r|r|r|}
\hline
\textbf{Proceso} & \textbf{Carga Inicial} & \textbf{Incremental} & \textbf{Registros/min} \\
\hline
Carga DimTiempo & 1.2 min & N/A & 4,870 \\
Carga dimensiones maestras & 7.5 min & 2.1 min & 1,532 \\
Carga FactFacturas (1 año) & 42.3 min & 4.2 min & 2,015 \\
Carga FactPagos (1 año) & 18.7 min & 2.5 min & 767 \\
Preparación Dataset ML & 8.2 min & 8.1 min & 69 \\
Validaciones de calidad & 6.8 min & 1.8 min & - \\
\hline
\textbf{Total} & \textbf{84.7 min} & \textbf{18.7 min} & - \\
\hline
\end{tabular}
\caption{Tiempos de ejecución del proceso ETL}
\end{table}

Los resultados muestran que el sistema procesa aproximadamente 1,200 a 2,000 registros por minuto en las operaciones más intensivas (carga de hechos), mientras que la carga incremental diaria se completa en menos de 20 minutos, haciendo viable la actualización nocturna automatizada sin impactar las operaciones diurnas.
\section{Análisis Exploratorio de Datos}
\subsection{Caracterización de la cartera crediticia}
El análisis de la cartera crediticia de la empresa mayorista revela las siguientes características principales:

\begin{itemize}
    \item \textbf{Distribución geográfica:} El 42\% de los clientes se concentra en la región Costa, 38\% en la Sierra, 15\% en el Oriente y 5\% en la región Insular.
    
    \item \textbf{Segmentación por volumen:} El 20\% de los clientes representa el 65\% del volumen total de la cartera, evidenciando una concentración significativa.
    
    \item \textbf{Antigüedad promedio:} La media de antigüedad como cliente es de 3.2 años, con una desviación estándar de 1.8 años.
    
    \item \textbf{Comportamiento de pago:} El tiempo promedio de pago es de 45 días, siendo el plazo estándar establecido de 30 días.
\end{itemize}

\subsection{Identificación de patrones y relaciones}
El análisis de correlaciones entre variables y comportamiento de pago identifica las siguientes relaciones significativas:

\begin{table}[ht]
\centering
\begin{tabular}{|p{6cm}|p{3cm}|p{6cm}|}
\hline
\textbf{Variable} & \textbf{Coeficiente de correlación} & \textbf{Significancia} \\
\hline
Máximo\_Días\_Atraso & 0.72 & Alta correlación positiva con probabilidad de morosidad futura \\
\hline
Variabilidad\_Pago & 0.68 & Alta correlación positiva con probabilidad de morosidad \\
\hline
Índice\_Estacionalidad & 0.61 & Correlación positiva, indicando mayor riesgo en negocios con alta estacionalidad \\
\hline
Antigüedad & -0.58 & Correlación negativa, sugiriendo menor riesgo en clientes más antiguos \\
\hline
Índice\_Concentración\_Producto & -0.43 & Correlación negativa, indicando menor riesgo en clientes diversificados \\
\hline
\end{tabular}
\caption{Principales correlaciones identificadas}
\end{table}
\newpage
El análisis de series temporales revela patrones estacionales significativos:

\begin{itemize}
    \item Incremento de morosidad en períodos post-festivos (enero-febrero y septiembre)
    \item Mayor cumplimiento en meses previos a temporadas altas (noviembre-diciembre)
    \item Variaciones regionales sincronizadas con ciclos económicos locales
\end{itemize}
\subsection{Análisis exploratorio de morosidad}

El Data Warehouse implementado permite realizar análisis exhaustivos del comportamiento de morosidad desde múltiples perspectivas. Los resultados del análisis exploratorio inicial revelan patrones significativos que posteriormente informaron el desarrollo del modelo predictivo.

\subsubsection{Distribución general de morosidad}

Del total de 85,242 facturas registradas en el período analizado, 14,573 facturas (17.1 por ciento) presentan algún grado de morosidad en el pago. El saldo pendiente total asciende a \$2,847,356 dólares, de los cuales \$487,923 dólares (17.1 por ciento) corresponden a facturas en mora.

\begin{table}[ht]
\centering
\begin{tabular}{|l|r|r|r|}
\hline
\textbf{Rango de Mora} & \textbf{Facturas} & \textbf{Saldo (\$)} & \textbf{\% Saldo} \\
\hline
Al Día & 70,669 & 2,359,433 & 82.9 \\
1-30 días & 6,234 & 189,567 & 6.7 \\
31-60 días & 3,891 & 134,782 & 4.7 \\
61-90 días & 2,045 & 78,934 & 2.8 \\
91-120 días & 1,234 & 42,156 & 1.5 \\
>120 días & 1,169 & 42,484 & 1.5 \\
\hline
\textbf{Total} & \textbf{85,242} & \textbf{2,847,356} & \textbf{100.0} \\
\hline
\end{tabular}
\caption{Distribución de facturas por rango de morosidad}
\end{table}

La distribución muestra que la mayor parte de la morosidad se concentra en los primeros 30 días de retraso (6.7 por ciento del saldo total), mientras que los casos de morosidad severa (mayor a 90 días) representan apenas el 3 por ciento del saldo pendiente. Este patrón sugiere que intervenciones tempranas en el primer mes de mora podrían tener un impacto significativo en la reducción de la cartera vencida.

\subsubsection{Análisis por dimensión geográfica}

El análisis de morosidad por región geográfica revela diferencias significativas en el comportamiento crediticio:

\begin{table}[ht]
\centering
\begin{tabular}{|l|r|r|r|r|}
\hline
\textbf{Regional} & \textbf{Clientes} & \textbf{Morosos} & \textbf{\% Morosos} & \textbf{Saldo Moroso} \\
\hline
Costa Norte & 287 & 62 & 21.6 & \$147,234 \\
Costa Sur & 412 & 58 & 14.1 & \$112,456 \\
Sierra Norte & 198 & 41 & 20.7 & \$89,123 \\
Sierra Centro & 345 & 52 & 15.1 & \$98,765 \\
Sierra Sur & 165 & 28 & 17.0 & \$40,345 \\
\hline
\textbf{Total} & \textbf{1,407} & \textbf{241} & \textbf{17.1} & \textbf{\$487,923} \\
\hline
\end{tabular}
\caption{Distribución de morosidad por región geográfica}
\end{table}

Las regiones Costa Norte y Sierra Norte presentan las tasas más altas de morosidad (21.6 por ciento y 20.7 por ciento respectivamente), mientras que Costa Sur mantiene la tasa más baja (14.1 por ciento). Estas diferencias regionales son estadísticamente significativas (p < 0.05) y fueron incorporadas como features en el modelo predictivo.

\subsubsection{Análisis temporal}

La evolución temporal de la morosidad durante el período de estudio muestra una tendencia decreciente hasta 2022, seguida de un incremento gradual:

\begin{table}[ht]
\centering
\begin{tabular}{|r|r|r|r|}
\hline
\textbf{Año} & \textbf{Tasa Morosidad (\%)} & \textbf{Facturas} & \textbf{Saldo Moroso (\$)} \\
\hline
2017 & 19.8 & 8,234 & \$78,456 \\
2018 & 18.5 & 9,567 & \$82,123 \\
2019 & 17.2 & 10,234 & \$75,678 \\
2020 & 22.3 & 9,012 & \$91,234 \\
2021 & 18.9 & 11,456 & \$85,456 \\
2022 & 15.4 & 12,345 & \$72,345 \\
2023 & 16.8 & 13,234 & \$78,234 \\
2024 & 17.1 & 11,160 & \$24,397 \\
\hline
\end{tabular}
\caption{Evolución temporal de la morosidad}
\end{table}

El pico de morosidad en 2020 (22.3 por ciento) coincide con el impacto económico de la pandemia COVID-19 en Ecuador. La recuperación gradual observada en 2021-2022 fue interrumpida por un repunte en 2023, lo que motivó la necesidad de implementar herramientas predictivas de gestión de riesgo crediticio.
\subsection{Segmentación de la cartera}
Mediante el algoritmo K-Means se identificaron cuatro segmentos principales en la cartera:

\begin{table}[ht]
\centering
\begin{tabular}{|p{2.5cm}|p{2.5cm}|p{9cm}|}
\hline
\textbf{Segmento} & \textbf{Proporción} & \textbf{Características principales} \\
\hline
Premium & 18\% & Alto volumen, baja morosidad, alta antigüedad, diversificación de productos \\
\hline
Estable & 35\% & Volumen medio, comportamiento predecible, morosidad ocasional de corto plazo \\
\hline
Estacional & 27\% & Alta variabilidad temporal, concentración en categorías específicas, morosidad cíclica \\
\hline
Alto riesgo & 20\% & Historial irregular, alta frecuencia de incumplimientos, baja antigüedad \\
\hline
\end{tabular}
\caption{Segmentación de cartera mediante K-Means}
\end{table}

Esta segmentación proporciona una base estructurada para la aplicación diferenciada de modelos predictivos y estrategias de gestión de riesgo.

\section{Desarrollo del Modelo Predictivo}
\subsection{Evaluación comparativa de algoritmos}
Se implementaron y evaluaron cinco algoritmos predictivos, con los siguientes resultados:

\begin{table}[ht]
\centering
\begin{tabular}{|p{3cm}|p{1.5cm}|p{1.5cm}|p{1.5cm}|p{1.5cm}|p{1.5cm}|}
\hline
\textbf{Algoritmo} & \textbf{Precisión} & \textbf{Recall} & \textbf{F1-Score} & \textbf{AUC} & \textbf{Tiempo (s)} \\
\hline
Regresión Logística & 0.76 & 0.71 & 0.73 & 0.82 & 1.2 \\
\hline
Random Forest & 0.83 & 0.79 & 0.81 & 0.87 & 3.5 \\
\hline
XGBoost & 0.87 & 0.84 & 0.85 & 0.91 & 4.2 \\
\hline
SVM & 0.79 & 0.76 & 0.77 & 0.83 & 2.8 \\
\hline
Red Neuronal & 0.82 & 0.80 & 0.81 & 0.88 & 8.7 \\
\hline
\end{tabular}
\caption{Comparativa de rendimiento de algoritmos}
\end{table}

XGBoost presenta el mejor rendimiento global, con una precisión del 87\% y un área bajo la curva ROC de 0.91, superando el umbral objetivo establecido del 80\%.

\subsection{Análisis de importancia de características}
El análisis de importancia de características en el modelo XGBoost seleccionado revela que las variables más determinantes son:

\begin{figure}[ht]
\centering
\begin{tabular}{|p{6cm}|p{3cm}|}
\hline
\textbf{Variable} & \textbf{Importancia relativa} \\
\hline
Máximo\_Días\_Atraso & 18.3\% \\
\hline
Variabilidad\_Pago & 14.7\% \\
\hline
Índice\_Cumplimiento & 12.9\% \\
\hline
Índice\_Estacionalidad & 10.5\% \\
\hline
Región & 9.8\% \\
\hline
Antigüedad & 8.4\% \\
\hline
Tendencia\_Compra & 7.2\% \\
\hline
Ratio\_Pago\_Plazo & 6.8\% \\
\hline
Categoría\_Principal & 5.9\% \\
\hline
Otros factores combinados & 5.5\% \\
\hline
\end{tabular}
\caption{Importancia relativa de variables en el modelo XGBoost}
\end{figure}

Estos resultados confirman que, además de los factores históricos de comportamiento de pago (Máximo\_Días\_Atraso, Índice\_Cumplimiento), las variables que capturan la estacionalidad y la ubicación geográfica tienen un impacto significativo en la predicción de morosidad.

\subsection{Validación cruzada y análisis de robustez}
La validación cruzada mediante 5-fold confirma la consistencia del rendimiento del modelo XGBoost:

\begin{table}[ht]
\centering
\begin{tabular}{|p{2cm}|p{2cm}|p{2cm}|p{2cm}|p{2cm}|}
\hline
\textbf{Fold} & \textbf{Precisión} & \textbf{Recall} & \textbf{F1-Score} & \textbf{AUC} \\
\hline
1 & 0.86 & 0.83 & 0.84 & 0.90 \\
\hline
2 & 0.88 & 0.85 & 0.86 & 0.92 \\
\hline
3 & 0.85 & 0.82 & 0.83 & 0.89 \\
\hline
4 & 0.87 & 0.84 & 0.85 & 0.91 \\
\hline
5 & 0.89 & 0.86 & 0.87 & 0.92 \\
\hline
\textbf{Media} & \textbf{0.87} & \textbf{0.84} & \textbf{0.85} & \textbf{0.91} \\
\hline
\textbf{Desv. Est.} & \textbf{0.015} & \textbf{0.014} & \textbf{0.014} & \textbf{0.012} \\
\hline
\end{tabular}
\caption{Resultados de validación cruzada 5-fold para XGBoost}
\end{table}

La baja desviación estándar entre folds indica alta robustez del modelo, lo que sugiere que su capacidad predictiva se mantendrá estable en nuevos datos.

\subsection{Matriz de confusión y análisis de errores}
La matriz de confusión del modelo final revela:

\begin{table}[ht]
\centering
\begin{tabular}{|p{3cm}|p{3cm}|p{3cm}|}
\hline
\textbf{n=72} & \textbf{Predicción: No moroso} & \textbf{Predicción: Moroso} \\
\hline
\textbf{Real: No moroso} & 45 (Verdaderos Negativos) & 5 (Falsos Positivos) \\
\hline
\textbf{Real: Moroso} & 4 (Falsos Negativos) & 18 (Verdaderos Positivos) \\
\hline
\end{tabular}
\caption{Matriz de confusión del modelo XGBoost en conjunto de prueba}
\end{table}

El análisis de los casos incorrectamente clasificados revela patrones específicos:

\begin{itemize}
    \item \textbf{Falsos positivos:} Predominantemente clientes con patrones estacionales extremos pero con cumplimiento eventual.
    
    \item \textbf{Falsos negativos:} Principalmente clientes con historiales estables que experimentaron cambios bruscos en su comportamiento debido a factores externos no capturados completamente en los datos (como cambios repentinos en condiciones de mercado local).
\end{itemize}

\section{Implementación del Sistema de Información}
\subsection{Arquitectura del sistema}

El sistema implementado se estructura en una arquitectura distribuida de tres capas que garantiza escalabilidad, mantenibilidad y rendimiento óptimo para el procesamiento de predicciones de morosidad.

\subsubsection{Detalle de la arquitectura implementada}

\begin{figure}[ht]
\centering
\begin{tabular}{|p{3cm}|p{4cm}|p{7cm}|}
\hline
\textbf{Capa} & \textbf{Componentes} & \textbf{Funcionalidades} \\
\hline
\multirow{4}{*}{Datos} & Base de datos SQL Server 2022 & Almacenamiento transaccional principal \\
\cline{2-3}
& Data Warehouse (DW\_Comisaseo) & Repositorio centralizado para análisis \\
\cline{2-3}
& Procesos ETL (SSIS) & Extracción y transformación automatizada \\
\cline{2-3}
& Sistema de respaldo & Backup incremental diario \\
\hline
\multirow{4}{*}{Procesamiento} & Servicio de predicción Python & Motor de machine learning con XGBoost \\
\cline{2-3}
& Programador de tareas (Cron) & Reentrenamiento automático mensual \\
\cline{2-3}
& API REST (Flask) & Interfaces para integración empresarial \\
\cline{2-3}
& Cache Redis & Almacenamiento temporal de predicciones \\
\hline
\multirow{3}{*}{Presentación} & Dashboard Power BI & Visualización ejecutiva y operativa \\
\cline{2-3}
& Sistema de alertas (SMTP) & Notificaciones automáticas por email \\
\cline{2-3}
& Reportes automáticos & Informes programados semanales \\
\hline
\end{tabular}
\caption{Arquitectura detallada del sistema implementado}
\end{figure}

\subsubsection{Patrones arquitectónicos aplicados}

\begin{itemize}
    \item \textbf{Patrón Repository:} Abstracción del acceso a datos mediante clases especializadas
    \item \textbf{Patrón Observer:} Sistema de notificaciones basado en eventos de predicción
    \item \textbf{Patrón Factory:} Creación dinámica de modelos según segmento de cliente
    \item \textbf{Patrón Singleton:} Gestión única de conexiones a base de datos
\end{itemize}

\subsubsection{Escalabilidad y rendimiento}

El diseño arquitectónico contempla:

\begin{enumerate}
    \item \textbf{Escalabilidad horizontal:} Posibilidad de añadir servidores de procesamiento
    \item \textbf{Balanceador de carga:} Nginx para distribuir requests de API
    \item \textbf{Cache distribuido:} Redis para reducir latencia en consultas frecuentes
    \item \textbf{Procesamiento asíncrono:} Celery para tareas de larga duración
\end{enumerate}

\subsubsection{Monitoreo y logging}

\begin{itemize}
    \item \textbf{Métricas de sistema:} CPU, memoria, espacio en disco
    \item \textbf{Métricas de aplicación:} Tiempo de respuesta de predicciones, precisión del modelo
    \item \textbf{Alertas proactivas:} Notificación ante degradación de rendimiento
    \item \textbf{Logs centralizados:} ELK Stack para análisis de logs
\end{itemize}

\# Sección 4.3.2 - Dashboard Interactivo (Detallado)


\subsection{Dashboard interactivo}

El dashboard interactivo constituye la interfaz principal para la visualización y gestión de las predicciones de morosidad, desarrollado en Power BI con integración directa a la base de datos SQL Server y diseñado para diferentes perfiles de usuario.

\subsubsection{Arquitectura del dashboard}

El dashboard se estructura en una arquitectura de tres niveles:

\begin{table}[ht]
\centering
\begin{tabular}{|p{3cm}|p{4cm}|p{7cm}|}
\hline
\textbf{Nivel} & \textbf{Componente} & \textbf{Descripción} \\
\hline
\multirow{2}{*}{Datos} & Gateway On-Premises & Conexión segura entre Power BI Service y SQL Server local \\
\cline{2-3}
& Modelo de datos & Esquema estrella optimizado para consultas analíticas \\
\hline
\multirow{3}{*}{Lógica} & Medidas DAX & Cálculos dinámicos de KPIs y métricas de riesgo \\
\cline{2-3}
& Filtros automáticos & Segmentación dinámica por región, fecha y segmento \\
\cline{2-3}
& Alertas condicionales & Formateo visual basado en umbrales de riesgo \\
\hline
\multirow{2}{*}{Presentación} & Visualizaciones interactivas & Gráficos, mapas y tablas con drill-down \\
\cline{2-3}
& Navegación contextual & Menús adaptativos según perfil de usuario \\
\hline
\end{tabular}
\caption{Arquitectura del dashboard de Power BI}
\end{table}

\subsubsection{Componentes principales del dashboard}

\textbf{1. Mapa de riesgo geográfico}
\begin{itemize}
    \item \textbf{Visualización:} Mapa coroplético de Ecuador con codificación por colores
    \item \textbf{Métricas mostradas:}
    \begin{itemize}
        \item Número de clientes por provincia
        \item Riesgo promedio por región (escala 0-100)
        \item Concentración de alertas rojas
        \item Tendencia de morosidad trimestral
    \end{itemize}
    \item \textbf{Interactividad:}
    \begin{itemize}
        \item Click en provincia para drill-down a nivel cantonal
        \item Tooltip con métricas detalladas por región
        \item Filtro cruzado con otros componentes
    \end{itemize}
    \item \textbf{Actualización:} Diaria a las 06:00 AM
\end{itemize}

\textbf{2. Panel de tendencias temporales}
\begin{itemize}
    \item \textbf{Visualización:} Gráfico de líneas múltiples con bandas de confianza
    \item \textbf{Series de tiempo:}
    \begin{itemize}
        \item Tasa de morosidad histórica vs predicha
        \item Volumen de cartera por mes
        \item Número de alertas generadas
        \item Efectividad de intervenciones
    \end{itemize}
    \item \textbf{Controles:}
    \begin{itemize}
        \item Selector de período (último mes, trimestre, año)
        \item Zoom temporal interactivo
        \item Comparación año sobre año
    \end{itemize}
    \item \textbf{Actualización:} Semanal, domingos a las 23:00
\end{itemize}

\textbf{3. Radar de segmentos de clientes}
\begin{itemize}
    \item \textbf{Visualización:} Gráfico radial multidimensional
    \item \textbf{Dimensiones evaluadas:}
    \begin{itemize}
        \item Volumen de cartera
        \item Nivel de riesgo promedio
        \item Rentabilidad del segmento
        \item Estabilidad de pagos
        \item Potencial de crecimiento
    \end{itemize}
    \item \textbf{Funcionalidades:}
    \begin{itemize}
        \item Comparación entre segmentos
        \item Selección de clientes específicos
        \item Export de datos del segmento seleccionado
    \end{itemize}
\end{itemize}

\textbf{4. Centro de alertas tempranas}
\begin{itemize}
    \item \textbf{Visualización:} Tabla dinámica con codificación de colores por prioridad
    \item \textbf{Información mostrada:}
    \begin{itemize}
        \item Cliente y datos de contacto
        \item Probabilidad de morosidad (\%)
        \item Nivel de alerta (Verde/Amarillo/Naranja/Rojo)
        \item Días para intervención recomendada
        \item Historial de acciones tomadas
        \item Responsable asignado
    \end{itemize}
    \item \textbf{Acciones disponibles:}
    \begin{itemize}
        \item Asignación de responsable
        \item Marcado como "en proceso"
        \item Agregar comentarios de seguimiento
        \item Generar reporte individual
    \end{itemize}
    \item \textbf{Actualización:} Tiempo real (cada 5 minutos)
\end{itemize}

\textbf{5. Simulador de escenarios}
\begin{itemize}
    \item \textbf{Funcionalidad:} Herramienta interactiva para evaluación de impacto
    \item \textbf{Variables ajustables:}
    \begin{itemize}
        \item Cambios en plazos de pago
        \item Modificaciones de cupos de crédito
        \item Implementación de descuentos por pronto pago
        \item Ajustes estacionales
    \end{itemize}
    \item \textbf{Resultados calculados:}
    \begin{itemize}
        \item Impacto en tasa de morosidad proyectada
        \item Efecto en flujo de caja
        \item Número de clientes afectados
        \item ROI estimado de la medida
    \end{itemize}
    \item \textbf{Casos de uso:}
    \begin{itemize}
        \item Planificación de políticas crediticias
        \item Evaluación de promociones comerciales
        \item Análisis de sensibilidad por segmento
    \end{itemize}
\end{itemize}

\subsubsection{Configuración por perfil de usuario}

\begin{table}[ht]
\centering
\begin{tabular}{|p{2.5cm}|p{2.5cm}|p{2.5cm}|p{3cm}|p{3cm}|}
\hline
\textbf{Componente} & \textbf{Tipo de visualización} & \textbf{Actualización} & \textbf{Interactividad} & \textbf{Usuarios objetivo} \\
\hline
Mapa de riesgo & Mapa coroplético con heat map & Diaria 06:00 & Filtros por región, drill-down, tooltips & Gerencia, Crédito, Regional \\
\hline
Panel tendencias & Series temporales múltiples & Semanal domingo 23:00 & Zoom, selección período, comparación & Gerencia, Financiero, Análisis \\
\hline
Radar clientes & Gráfico radial 5D & Diaria 06:00 & Selección segmentos, export datos & Crédito, Ventas, Marketing \\
\hline
Alertas tempranas & Tabla dinámica ordenable & Tiempo real (5 min) & Asignación, seguimiento, comentarios & Cobranzas, Supervisores \\
\hline
Simulador & Controles deslizantes & Bajo demanda & Ajuste parámetros, cálculo dinámico & Gerencia, Crédito, Finanzas \\
\hline
\end{tabular}
\caption{Configuración detallada por componente del dashboard}
\end{table}

\subsubsection{Métricas de rendimiento del dashboard}

Durante el período piloto se registraron las siguientes métricas de uso:

\begin{itemize}
    \item \textbf{Usuarios activos diarios:} 12 promedio (rango 8-18)
    \item \textbf{Sesiones promedio por usuario:} 3.2 diarias
    \item \textbf{Tiempo promedio de sesión:} 14.7 minutos
    \item \textbf{Componente más utilizado:} Centro de alertas (78\% del tiempo)
    \item \textbf{Tiempo de carga inicial:} 2.8 segundos promedio
    \item \textbf{Disponibilidad del dashboard:} 99.2\%
    \item \textbf{Errores de actualización:} 0.6\% de las programadas
\end{itemize}

\subsubsection{Personalización y configuración avanzada}

\textbf{Temas visuales personalizados}
\begin{itemize}
    \item Paleta de colores corporativa de la empresa
    \item Iconografía consistente con identidad visual
    \item Tipografía optimizada para legibilidad en pantallas
\end{itemize}

\textbf{Configuración de alertas por usuario}
\begin{itemize}
    \item Umbrales personalizables por departamento
    \item Notificaciones push configurables
    \item Frecuencia de reportes ajustable
    \item Filtros predeterminados por rol
\end{itemize}

\textbf{Integración con sistemas empresariales}
\begin{itemize}
    \item Conexión directa con ERP para datos transaccionales
    \item Sincronización con CRM para información de contacto
    \item Export automático a Excel para análisis offline
    \item Integración con sistema de tickets para seguimiento
\end{itemize}

\subsubsection{Medidas DAX implementadas}

Las siguientes medidas DAX fueron desarrolladas para cálculos dinámicos en el dashboard:

\begin{verbatim}
// Tasa de Morosidad Actual
Tasa_Morosidad = 
DIVIDE(
    CALCULATE(
        COUNTROWS(Predicciones),
        Predicciones[Nivel_Alerta] IN {"Naranja", "Rojo"}
    ),
    COUNTROWS(Predicciones),
    0
)

// Predicciones de Alto Riesgo
Alto_Riesgo_Count = 
CALCULATE(
    COUNTROWS(Predicciones),
    Predicciones[Probabilidad_Morosidad] > 0.75
)

// Efectividad de Intervenciones
Efectividad_Intervencion = 
VAR ClientesIntervenidos = 
    CALCULATE(
        COUNTROWS(Seguimientos),
        Seguimientos[Accion_Tomada] <> BLANK()
    )
VAR ClientesRecuperados = 
    CALCULATE(
        COUNTROWS(Seguimientos),
        Seguimientos[Estado_Final] = "Normalizado"
    )
RETURN
    DIVIDE(ClientesRecuperados, ClientesIntervenidos, 0)

// Tendencia de Riesgo (3 meses)
Tendencia_Riesgo = 
VAR RiesgoActual = [Tasa_Morosidad]
VAR RiesgoAnterior = 
    CALCULATE(
        [Tasa_Morosidad],
        DATEADD(Predicciones[Fecha_Prediccion], -3, MONTH)
    )
RETURN
    RiesgoActual - RiesgoAnterior
\end{verbatim}

\subsubsection{Configuración de seguridad}

\textbf{Row Level Security (RLS)}
\begin{itemize}
    \item Filtros automáticos por región según perfil de usuario
    \item Acceso restringido a datos de ciertos segmentos
    \item Ocultación de información financiera sensible
    \item Logs de acceso por usuario y timestamp
\end{itemize}

\textbf{Roles de seguridad definidos}
\begin{table}[ht]
\centering
\begin{tabular}{|p{3cm}|p{5cm}|p{6cm}|}
\hline
\textbf{Rol} & \textbf{Acceso permitido} & \textbf{Restricciones} \\
\hline
Gerencia & Todos los datos y componentes & Ninguna \\
\hline
Supervisor Crédito & Datos de su región asignada & Solo clientes de su zona \\
\hline
Analista Cobranzas & Alertas y seguimientos & Sin acceso a simulador \\
\hline
Consulta & Dashboards en modo lectura & Sin modificación de parámetros \\
\hline
Invitado & Resumen ejecutivo únicamente & Datos agregados sin detalle \\
\hline
\end{tabular}
\caption{Matriz de roles y permisos del dashboard}
\end{table}

\subsubsection{Optimizaciones de rendimiento implementadas}

\begin{enumerate}
    \item \textbf{Modelado de datos optimizado:}
    \begin{itemize}
        \item Esquema estrella con tablas de dimensiones desnormalizadas
        \item Índices columnares en tablas de hechos
        \item Particionamiento por fecha en tabla de predicciones
        \item Compresión de datos históricos mayores a 12 meses
    \end{itemize}
    
    \item \textbf{Estrategias de cache:}
    \begin{itemize}
        \item Cache de consultas DAX frecuentes
        \item Actualización incremental de datasets
        \item Pre-agregaciones para métricas comunes
        \item Cache de visualizaciones estáticas
    \end{itemize}
    
    \item \textbf{Optimización de consultas:}
    \begin{itemize}
        \item Uso de variables DAX para evitar recálculos
        \item Filtros tempranos en medidas complejas
        \item Eliminación de relaciones bidireccionales innecesarias
        \item Optimización de cardinalidad en relaciones
    \end{itemize}
\end{enumerate}

\subsubsection{Métricas de adopción y satisfacción}

\textbf{Adopción por departamento}
\begin{itemize}
    \item \textbf{Cobranzas:} 100\% adopción, uso diario del centro de alertas
    \item \textbf{Crédito:} 95\% adopción, uso frecuente del simulador
    \item \textbf{Gerencia:} 85\% adopción, revisión semanal de tendencias
    \item \textbf{Ventas:} 70\% adopción, consulta del radar de clientes
\end{itemize}

\textbf{Feedback de usuarios (escala 1-5)}
\begin{table}[ht]
\centering
\begin{tabular}{|p{4cm}|p{2cm}|p{2cm}|p{2cm}|p{2cm}|}
\hline
\textbf{Aspecto evaluado} & \textbf{Gerencia} & \textbf{Crédito} & \textbf{Cobranzas} & \textbf{Promedio} \\
\hline
Facilidad de uso & 4.3 & 4.1 & 4.5 & 4.3 \\
\hline
Velocidad de respuesta & 4.0 & 4.2 & 4.4 & 4.2 \\
\hline
Relevancia de información & 4.7 & 4.6 & 4.8 & 4.7 \\
\hline
Diseño visual & 4.2 & 4.0 & 4.1 & 4.1 \\
\hline
Funcionalidad móvil & 3.8 & 3.9 & 4.0 & 3.9 \\
\hline
\textbf{Satisfacción general} & \textbf{4.2} & \textbf{4.2} & \textbf{4.4} & \textbf{4.3} \\
\hline
\end{tabular}
\caption{Evaluación de satisfacción del dashboard por departamento}
\end{table}
```

\subsection{Sistema de alertas tempranas}
El sistema de alertas tempranas implementa un enfoque estratificado:

\begin{table}[ht]
\centering
\begin{tabular}{|p{2.5cm}|p{2.5cm}|p{3cm}|p{6cm}|}
\hline
\textbf{Nivel de alerta} & \textbf{Criterio} & \textbf{Tiempo anticipación} & \textbf{Acciones recomendadas} \\
\hline
Verde & Prob. < 20\% & Monitoreo & Seguimiento regular según política estándar \\
\hline
Amarillo & Prob. 20-50\% & 30 días & Contacto preventivo, verificación situación \\
\hline
Naranja & Prob. 50-75\% & 15 días & Contacto prioritario, planes de pago especiales \\
\hline
Rojo & Prob. > 75\% & 7 días & Restricción preventiva, contacto gerencial \\
\hline
\end{tabular}
\caption{Niveles del sistema de alertas tempranas}
\end{table}

\section{Validación y Evaluación de Impacto}
\# Sección 4.4.1 - Casos de Prueba y Resultados Detallados

```latex

\subsection{Pruebas piloto y resultados iniciales}

El sistema fue sometido a un proceso riguroso de pruebas durante un período piloto de 3 meses (enero-marzo 2025), implementando diferentes casos de prueba para validar la funcionalidad, precisión y rendimiento del modelo predictivo.

\subsubsection{Definición de casos de prueba}

Se establecieron los siguientes casos de prueba principales:

\begin{table}[ht]
\centering
\begin{tabular}{|p{1.5cm}|p{4cm}|p{3cm}|p{5cm}|}
\hline
\textbf{ID} & \textbf{Descripción} & \textbf{Objetivo} & \textbf{Criterio de Éxito} \\
\hline
CP-001 & Predicción de clientes nuevos & Validar capacidad predictiva para clientes con historial < 6 meses & Precisión > 75\% \\
\hline
CP-002 & Predicción estacional & Evaluar comportamiento durante picos estacionales & Mantener precisión > 80\% \\
\hline
CP-003 & Segmentación automática & Verificar clasificación correcta por segmentos & 95\% clientes correctamente clasificados \\
\hline
CP-004 & Alertas tempranas & Validar sistema de notificaciones & 100\% alertas enviadas en < 5 min \\
\hline
CP-005 & Carga de datos masiva & Probar rendimiento con volúmenes altos & Procesar 1000 registros en < 10 min \\
\hline
CP-006 & Integración con sistemas & Verificar conectividad con ERP & 99.9\% disponibilidad \\
\hline
CP-007 & Dashboard en tiempo real & Validar actualización de visualizaciones & Refresh automático cada 4 horas \\
\hline
CP-008 & Recuperación ante fallos & Probar continuidad del servicio & Recuperación en < 15 min \\
\hline
\end{tabular}
\caption{Casos de prueba definidos para validación del sistema}
\end{table}

\subsubsection{Resultados detallados por caso de prueba}

\textbf{CP-001: Predicción de clientes nuevos}
\begin{itemize}
    \item \textbf{Muestra:} 28 clientes nuevos ingresados durante el piloto
    \item \textbf{Resultado:} Precisión del 78.6\% (22 de 28 predicciones correctas)
    \item \textbf{Análisis:} Superó el umbral mínimo del 75\%, validando la capacidad del modelo para clientes con historial limitado
    \item \textbf{Observaciones:} Mayor dificultad en clientes del segmento "Estacional" (60\% precisión vs 85\% en otros segmentos)
\end{itemize}

\textbf{CP-002: Predicción estacional}
\begin{itemize}
    \item \textbf{Período evaluado:} Temporada navideña 2024 y post-navideña enero 2025
    \item \textbf{Muestra:} 156 predicciones durante picos estacionales
    \item \textbf{Resultado:} Precisión del 82.1\%
    \item \textbf{Análisis:} Mantuvo rendimiento superior al 80\% incluso en períodos de alta variabilidad
    \item \textbf{Factor clave:} Variables de estacionalidad representaron 15\% del poder predictivo
\end{itemize}

\textbf{CP-003: Segmentación automática}
\begin{itemize}
    \item \textbf{Muestra:} 238 clientes completos de la cartera
    \item \textbf{Resultado:} 96.2\% clientes correctamente clasificados (229 de 238)
    \item \textbf{Errores:} 9 clientes reclasificados manualmente por analistas
    \item \textbf{Distribución final:}
    \begin{itemize}
        \item Premium: 43 clientes (18.1\%)
        \item Estable: 83 clientes (34.9\%)
        \item Estacional: 64 clientes (26.9\%)
        \item Alto riesgo: 48 clientes (20.2\%)
    \end{itemize}
\end{itemize}

\textbf{CP-004: Alertas tempranas}
\begin{itemize}
    \item \textbf{Alertas generadas:} 62 en total durante el piloto
    \item \textbf{Distribución por nivel:}
    \begin{itemize}
        \item Rojas: 15 (24.2\%)
        \item Naranjas: 23 (37.1\%)
        \item Amarillas: 24 (38.7\%)
    \end{itemize}
    \item \textbf{Tiempo promedio de envío:} 2.3 minutos
    \item \textbf{Tasa de entrega exitosa:} 100\%
    \item \textbf{Falsos positivos:} 6 alertas (9.7\%)
\end{itemize}

\textbf{CP-005: Carga de datos masiva}
\begin{itemize}
    \item \textbf{Volumen procesado:} 1,500 registros de prueba
    \item \textbf{Tiempo de procesamiento:} 7.2 minutos
    \item \textbf{Memoria utilizada:} Pico de 2.1 GB
    \item \textbf{CPU promedio:} 68\% durante procesamiento
    \item \textbf{Resultado:} Cumplió objetivo de < 10 minutos
\end{itemize}

\textbf{CP-006: Integración con sistemas}
\begin{itemize}
    \item \textbf{Período de monitoreo:} 90 días continuos
    \item \textbf{Disponibilidad del servicio:} 99.94\%
    \item \textbf{Tiempo de respuesta promedio API:} 1.2 segundos
    \item \textbf{Interrupciones:} 2 incidentes menores (< 30 min cada uno)
    \item \textbf{Transacciones exitosas:} 99.97\%
\end{itemize}

\textbf{CP-007: Dashboard en tiempo real}
\begin{itemize}
    \item \textbf{Actualizaciones programadas:} 540 durante el piloto (6 por día)
    \item \textbf{Actualizaciones exitosas:} 537 (99.4\%)
    \item \textbf{Tiempo promedio de actualización:} 3.1 minutos
    \item \textbf{Usuarios activos promedio:} 12 por día
    \item \textbf{Tiempo de carga promedio:} 2.8 segundos
\end{itemize}

\textbf{CP-008: Recuperación ante fallos}
\begin{itemize}
    \item \textbf{Simulaciones realizadas:} 5 escenarios de falla
    \item \textbf{Tiempo promedio de detección:} 3.2 minutos
    \item \textbf{Tiempo promedio de recuperación:} 11.8 minutos
    \item \textbf{Pérdida de datos:} 0 registros en todos los casos
    \item \textbf{Procedimientos activados:} Backup automático y failover
\end{itemize}

\subsubsection{Métricas consolidadas del piloto}

\begin{table}[ht]
\centering
\begin{tabular}{|p{5cm}|p{3cm}|p{3cm}|p{3cm}|}
\hline
\textbf{Métrica} & \textbf{Objetivo} & \textbf{Resultado} & \textbf{Estado} \\
\hline
Precisión general del modelo & > 80\% & 84.0\% & \textcolor{green}{\checkmark  Cumplido} \\
\hline
Tiempo de anticipación promedio & > 10 días & 15.2 días & \textcolor{green}{\checkmark Cumplido} \\
\hline
Reducción de morosidad & > 15\% & 18.0\% & \textcolor{green}{\checkmark Cumplido} \\
\hline
Disponibilidad del sistema & > 99\% & 99.94\% & \textcolor{green}{\checkmark Cumplido} \\
\hline
Satisfacción de usuarios & > 4.0/5 & 4.3/5 & \textcolor{green}{\checkmark Cumplido} \\
\hline
ROI proyectado & < 12 meses & 6 meses & \textcolor{green}{\checkmark Cumplido} \\
\hline
\end{tabular}
\caption{Métricas consolidadas del período piloto}
\end{table}

\subsubsection{Análisis de desviaciones y ajustes realizados}

Durante el piloto se identificaron las siguientes desviaciones y se implementaron los ajustes correspondientes:

\begin{enumerate}
    \item \textbf{Baja precisión en segmento "Estacional":}
    \begin{itemize}
        \item \textbf{Problema:} Precisión del 76\% vs 84\% objetivo
        \item \textbf{Causa:} Variables estacionales insuficientes para capturar patrones complejos
        \item \textbf{Ajuste:} Incorporación de 3 variables adicionales de tendencia mensual
        \item \textbf{Resultado:} Mejora a 79\% de precisión
    \end{itemize}
    
    \item \textbf{Latencia en actualizaciones de dashboard:}
    \begin{itemize}
        \item \textbf{Problema:} Tiempo de actualización > 5 minutos en 15\% de casos
        \item \textbf{Causa:} Consultas no optimizadas en Power BI
        \item \textbf{Ajuste:} Optimización de consultas DAX y creación de vistas materializadas
        \item \textbf{Resultado:} Reducción a 3.1 minutos promedio
    \end{itemize}
    
    \item \textbf{Falsos positivos en alertas:}
    \begin{itemize}
        \item \textbf{Problema:} 9.7\% de alertas fueron falsos positivos
        \item \textbf{Causa:} Umbrales muy sensibles para ciertos segmentos
        \item \textbf{Ajuste:} Calibración de umbrales específicos por segmento
        \item \textbf{Resultado:} Reducción a 6.2\% de falsos positivos
    \end{itemize}
\end{enumerate}

\subsubsection{Lecciones aprendidas}

\begin{itemize}
    \item La segmentación previa mejora significativamente la precisión del modelo
    \item Las variables geográficas tienen mayor impacto del esperado inicialmente
    \item Es necesario mantener umbrales diferenciados por segmento de cliente
    \item El monitoreo continuo es esencial para detectar degradación temprana
    \item La capacitación de usuarios finales acelera la adopción del sistema
\end{itemize}


\subsection{Comparación con método tradicional}
La comparación con el método tradicional previamente utilizado muestra mejoras significativas:

\begin{table}[ht]
\centering
\begin{tabular}{|p{4cm}|p{3cm}|p{3cm}|p{3cm}|}
\hline
\textbf{Indicador} & \textbf{Método tradicional} & \textbf{Modelo predictivo} & \textbf{Mejora} \\
\hline
Precisión & 61\% & 84\% & +23\% \\
\hline
Tiempo anticipación & 3 días & 15 días (promedio) & +400\% \\
\hline
Falsos positivos & 28\% & 10\% & -64\% \\
\hline
Cobertura (recall) & 58\% & 82\% & +41\% \\
\hline
Tiempo análisis & 45 min/cliente & 2 min/cliente & -96\% \\
\hline
\end{tabular}
\caption{Comparativa entre método tradicional y modelo predictivo}
\end{table}

\subsection{Análisis costo-beneficio}
El análisis económico de la implementación revela:

\begin{itemize}
    \item \textbf{Costos de implementación:} \$12,500 (desarrollo, infraestructura, capacitación)
    
    \item \textbf{Costos operativos mensuales:} \$850 (mantenimiento, licencias, soporte)
    
    \item \textbf{Beneficios directos:} Reducción estimada de pérdidas por incobrables de \$4,200 mensuales
    
    \item \textbf{Beneficios indirectos:} Reducción de tiempo operativo valorada en \$1,800 mensuales
    
    \item \textbf{ROI proyectado:} 6 meses para recuperación de inversión inicial
\end{itemize}

\section{Discusión de los Resultados}
\subsection{Interpretación de hallazgos clave}
Los resultados obtenidos permiten extraer las siguientes conclusiones principales:

\begin{enumerate}
    \item \textbf{Predictores determinantes:} El análisis de importancia de variables confirma que la variabilidad en el comportamiento de pago es más predictiva que el simple historial de cumplimiento, lo que coincide con los hallazgos de \cite{torres2023inteligencia}.
    
    \item \textbf{Factores contextuales:} La significativa importancia de variables geográficas y estacionales (combinadas representan más del 20\% del poder predictivo) valida la hipótesis inicial sobre la relevancia de estos factores en el contexto mayorista ecuatoriano.
    
    \item \textbf{Efectividad diferenciada:} El modelo muestra mejor desempeño predictivo en los segmentos "Premium" y "Alto riesgo", mientras que presenta mayor dificultad en la predicción para el segmento "Estacional", lo que sugiere la necesidad de ajustes específicos para este grupo.
    
    \item \textbf{Anticipación efectiva:} El incremento en el tiempo de anticipación (de 3 a 15 días en promedio) representa una mejora sustancial en la capacidad de intervención preventiva.
\end{enumerate}

\subsection{Limitaciones del estudio}
A pesar de los resultados positivos, es importante reconocer las siguientes limitaciones:

\begin{itemize}
    \item \textbf{Tamaño de muestra:} Con 238 clientes, aunque suficiente para el análisis, representa una limitación para técnicas avanzadas como redes neuronales profundas.
    
    \item \textbf{Factores externos no capturados:} Variables macroeconómicas locales y eventos específicos de mercado no están completamente integrados en el modelo actual.
    
    \item \textbf{Temporalidad:} La validación se realizó en un periodo de 3 meses, lo que podría no capturar completamente los ciclos estacionales anuales.
    
    \item \textbf{Generalización:} El modelo está optimizado para el contexto específico de la empresa estudiada, lo que podría limitar su aplicabilidad directa a otras organizaciones del sector.
\end{itemize}

\subsection{Comparación con estudios previos}
Contrastando los resultados con la literatura existente:

\begin{itemize}
    \item La precisión obtenida (84\% en implementación real) supera el promedio reportado por \cite{garcia2024machine} para modelos similares en el sector comercial (75-80\%).
    
    \item La identificación de patrones estacionales como factores predictivos significativos coincide con los hallazgos de \cite{ramirez2023predictive} en mercados regionales.
    
    \item La efectividad de XGBoost como algoritmo óptimo confirma los resultados reportados por \cite{torres2023inteligencia} en problemas de clasificación crediticia.
    
    \item El enfoque de segmentación previa mediante K-Means antes de la aplicación de modelos predictivos muestra beneficios no reportados consistentemente en estudios anteriores.
\end{itemize}

\subsection{Implicaciones prácticas}
Los resultados del estudio tienen las siguientes implicaciones prácticas para la gestión crediticia en el sector mayorista:

\begin{enumerate}
    \item \textbf{Enfoque preventivo:} La transición de un modelo reactivo a uno predictivo permite intervenciones tempranas que benefician tanto a la empresa como a los clientes.
    
    \item \textbf{Personalización por segmento:} La identificación de segmentos con diferentes perfiles de riesgo sugiere la implementación de políticas crediticias diferenciadas.
    
    \item \textbf{Relevancia contextual:} La importancia de factores regionales y estacionales destaca la necesidad de considerar el contexto local en la evaluación crediticia.
    
    \item \textbf{Automatización eficiente:} La reducción significativa en tiempo de análisis (96\%) libera recursos para actividades de mayor valor agregado.
\end{enumerate}
\chapter{Conclusiones y Recomendaciones}

\section{Conclusiones}
A partir de los resultados obtenidos en el desarrollo e implementación del modelo predictivo para la identificación temprana del riesgo de morosidad, se derivan las siguientes conclusiones:

\begin{enumerate}
    \item El desarrollo del modelo predictivo basado en técnicas de machine learning ha permitido alcanzar una precisión del 84\% en la identificación temprana del riesgo de morosidad, superando significativamente el objetivo establecido del 80\% y mejorando en un 23\% la precisión del método tradicional anteriormente utilizado.
    
    \item La integración de variables geográficas y estacionales en el modelo predictivo ha demostrado ser fundamental, representando más del 20\% del poder predictivo total. Esto confirma la hipótesis inicial sobre la importancia de estos factores en el contexto del comercio mayorista ecuatoriano.
    
    \item La metodología CRISP-DM ha demostrado ser un marco efectivo para estructurar el proceso de desarrollo del modelo predictivo, permitiendo un enfoque sistemático desde la comprensión del negocio hasta la implementación y evaluación de resultados.
    
    \item La segmentación previa de la cartera mediante técnicas de clustering (K-Means) ha facilitado la identificación de cuatro perfiles distintivos de clientes (Premium, Estable, Estacional y Alto riesgo), lo que permite una aplicación más precisa y contextualizada de los modelos predictivos.
    
    \item El algoritmo XGBoost ha mostrado el mejor rendimiento entre las técnicas evaluadas, con métricas consistentemente superiores en precisión, sensibilidad, especificidad y área bajo la curva ROC.
    
    \item La implementación del sistema de alertas tempranas ha permitido ampliar el tiempo promedio de anticipación de 3 a 15 días, lo que ha facilitado intervenciones preventivas efectivas que resultaron en una reducción del 18\% en la tasa general de morosidad durante el periodo piloto.
    
    \item El análisis de importancia de características ha revelado que la variabilidad en el comportamiento de pago (representada por variables como Variabilidad\_Pago y Máximo\_Días\_Atraso) posee mayor poder predictivo que indicadores estáticos tradicionales, lo que sugiere la necesidad de un enfoque dinámico en la evaluación crediticia.
    
    \item El dashboard interactivo implementado ha facilitado la interpretación y utilización de las predicciones generadas por el modelo, permitiendo que usuarios con diferentes niveles de experticia técnica puedan beneficiarse de los resultados predictivos.
    
    \item El análisis costo-beneficio revela un retorno de inversión estimado de 6 meses, considerando tanto los beneficios directos (reducción de incobrables) como indirectos (optimización de procesos operativos).
    
    \item La reducción del 64\% en falsos positivos respecto al método tradicional representa una mejora significativa en la eficiencia operativa, al disminuir el tiempo y recursos dedicados a casos incorrectamente identificados como de alto riesgo.
\end{enumerate}

\section{Recomendaciones}

Con base en la experiencia adquirida durante el desarrollo e implementación del proyecto, así como en las limitaciones identificadas, se plantean las siguientes recomendaciones:

\begin{enumerate}
    \item \textbf{Expansión del conjunto de datos:} Incorporar progresivamente datos de nuevos clientes y transacciones para fortalecer la capacidad predictiva del modelo, especialmente para el segmento "Estacional" donde se identificaron mayores desafíos.
    
    \item \textbf{Integración de variables externas:} Complementar el modelo con indicadores macroeconómicos regionales y sectoriales que podrían capturar factores externos no considerados actualmente.
    
    \item \textbf{Refinamiento por segmento:} Desarrollar modelos específicos para cada uno de los segmentos identificados, optimizando los parámetros y variables según las características particulares de cada grupo.
    
    \item \textbf{Implementación de aprendizaje continuo:} Establecer un proceso automatizado de reentrenamiento periódico que incorpore nuevos datos y resultados de intervenciones, permitiendo que el modelo evolucione y se adapte a cambios en patrones crediticios.
    
    \item \textbf{Ampliación del periodo de validación:} Extender el periodo de validación a un ciclo anual completo para evaluar con mayor precisión el desempeño del modelo ante la estacionalidad característica del sector.
    
    \item \textbf{Desarrollo de interfaces de integración:} Implementar APIs adicionales que faciliten la integración del sistema predictivo con otras plataformas empresariales, como sistemas de gestión de relaciones con clientes (CRM) y planificación de recursos empresariales (ERP).
    
    \item \textbf{Capacitación avanzada a usuarios:} Desarrollar un programa de formación continua para los usuarios del sistema, abordando tanto aspectos técnicos de interpretación de resultados como estrategias de intervención basadas en las predicciones.
    
    \item \textbf{Exploración de técnicas explicativas:} Implementar métodos como SHAP (SHapley Additive exPlanations) o LIME (Local Interpretable Model-agnostic Explanations) para mejorar la interpretabilidad de las predicciones individuales, facilitando la explicación de resultados a clientes y stakeholders.
    
    \item \textbf{Estudios comparativos sectoriales:} Promover colaboraciones con otras empresas del sector para realizar benchmarking de indicadores de riesgo crediticio y validar la generalización del modelo en contextos similares.
    
    \item \textbf{Desarrollo de módulos predictivos complementarios:} Expandir el alcance del sistema con módulos adicionales enfocados en predecir el monto probable de recuperación y el tiempo estimado hasta la normalización en casos de morosidad.
\end{enumerate}

\section{Trabajo Futuro}

El presente proyecto establece las bases para diversas líneas de investigación y desarrollo futuro:

\begin{enumerate}
    \item \textbf{Modelos híbridos:} Explorar la combinación de diferentes técnicas de machine learning con enfoques de análisis de redes sociales y análisis de texto para incorporar datos no estructurados disponibles en interacciones con clientes.
    
    \item \textbf{Sistemas de recomendación:} Desarrollar algoritmos que no solo predigan el riesgo de morosidad, sino que recomienden estrategias específicas de intervención basadas en características del cliente y patrones históricos de efectividad.
    
    \item \textbf{Análisis predictivo multidimensional:} Expandir el enfoque predictivo para abordar simultáneamente múltiples aspectos del comportamiento del cliente, incluyendo patrones de compra, sensibilidad a promociones y potencial de crecimiento, creando un perfil integral que complemente la evaluación de riesgo.
    
    \item \textbf{Aplicación de aprendizaje por refuerzo:} Implementar técnicas de aprendizaje por refuerzo para optimizar dinámicamente las estrategias de intervención, permitiendo que el sistema aprenda de la efectividad de acciones previas.
    
    \item \textbf{Extensión a otros sectores:} Adaptar la metodología y arquitectura desarrolladas para su aplicación en otros sectores con dinámicas crediticias similares, como distribución mayorista de alimentos, material de construcción o equipamiento tecnológico.
    
    \item \textbf{Incorporación de tecnologías emergentes:} Explorar la integración de tecnologías como blockchain para mejorar la transparencia y trazabilidad de las transacciones crediticias, o el Internet de las Cosas (IoT) para capturar datos operativos relevantes de clientes mayoristas.
\end{enumerate}

\section{Contribuciones Principales}

Las contribuciones más significativas de este trabajo pueden resumirse en:

\begin{enumerate}
    \item El desarrollo de un modelo predictivo con alta precisión (84\%) adaptado específicamente a las particularidades del sector mayorista ecuatoriano, considerando factores regionales y estacionales frecuentemente omitidos en modelos genéricos.
    
    \item La validación empírica de la efectividad de técnicas avanzadas de machine learning como XGBoost en un contexto con restricciones de datos (cartera limitada de clientes) y alta heterogeneidad.
    
    \item La integración exitosa del modelo predictivo en la operación cotidiana de la empresa mediante un sistema de información completo que abarca desde la captura y procesamiento de datos hasta la visualización interactiva y generación de alertas.
    
    \item La cuantificación del impacto económico y operativo derivado de la implementación de técnicas de inteligencia de negocios, demostrando un retorno de inversión tangible y contribuyendo a la justificación de inversiones similares en el sector.
    
    \item La documentación metodológica detallada del proceso de desarrollo, que puede servir como referencia para proyectos similares en el contexto de empresas mayoristas ecuatorianas y latinoamericanas.
\end{enumerate}

Este trabajo demuestra que la aplicación de técnicas avanzadas de machine learning y análisis predictivo puede generar beneficios significativos incluso en contextos empresariales con recursos tecnológicos limitados, siempre que se adapten adecuadamente a las particularidades del sector y se integren efectivamente en los procesos operativos existentes.

% ======================================
% BIBLIOGRAFÍA
% ======================================
\bibliographystyle{apalike}
\bibliography{bibliography}

% ======================================
% ANEXOS
% ======================================
\part*{Anexos}
\addcontentsline{toc}{part}{Anexos}
\chapter{Anexos}

\section{Anexo A: Código Fuente del Modelo Predictivo}

A continuación, se presenta el código principal para el entrenamiento del modelo XGBoost utilizado en el proyecto:

\begin{verbatim}
# Importación de librerías
import pandas as pd
import numpy as np
import pyodbc
from sklearn.model_selection import train_test_split, GridSearchCV
from sklearn.preprocessing import StandardScaler, OneHotEncoder
from sklearn.compose import ColumnTransformer
from sklearn.pipeline import Pipeline
from sklearn.metrics import classification_report, confusion_matrix, roc_auc_score
import xgboost as xgb
import matplotlib.pyplot as plt
import seaborn as sns

# Conexión a SQL Server 2022
def conectar_sql_server():
    server = 'servidor-sqlserver'
    database = 'DW_Comisaseo'
    username = 'usuario_analisis'
    password = 'password'
    
    connection_string = f'''
    DRIVER={{ODBC Driver 17 for SQL Server}};
    SERVER={server};
    DATABASE={database};
    UID={username};
    PWD={password}
    '''
    
    conn = pyodbc.connect(connection_string)
    return conn

# Cargar datos desde SQL Server
def cargar_datos():
    conn = conectar_sql_server()
    
    query = """
    SELECT 
        ID_Cliente,
        Region,
        Antiguedad_Meses,
        Promedio_Compra_Mensual,
        Frecuencia_Compra_Mensual,
        Dias_Promedio_Pago,
        Maximo_Dias_Atraso,
        Porcentaje_Devoluciones,
        Indice_Estacionalidad,
        Ratio_Pago_Plazo,
        Variabilidad_Pago,
        Indice_Concentracion_Producto,
        Tendencia_Compra,
        Indice_Cumplimiento,
        Moroso
    FROM V_Datos_ML
    """
    
    df = pd.read_sql(query, conn)
    conn.close()
    return df

# Cargar datos
df = cargar_datos()

# Definición de variables predictoras y objetivo
X = df.drop(['ID_Cliente', 'Moroso'], axis=1)
y = df['Moroso']

# División de conjunto de datos
X_train, X_test, y_train, y_test = train_test_split(
    X, y, test_size=0.3, random_state=42, stratify=y)

# Identificación de columnas por tipo
cat_cols = X.select_dtypes(include=['object', 'category']).columns.tolist()
num_cols = X.select_dtypes(include=['int64', 'float64']).columns.tolist()

# Definición de preprocesador
preprocessor = ColumnTransformer(
    transformers=[
        ('num', StandardScaler(), num_cols),
        ('cat', OneHotEncoder(handle_unknown='ignore'), cat_cols)
    ])

# Definición de pipeline con XGBoost
xgb_pipeline = Pipeline([
    ('preprocessor', preprocessor),
    ('classifier', xgb.XGBClassifier(objective='binary:logistic', random_state=42))
])

# Definición de parámetros para optimización
param_grid = {
    'classifier__n_estimators': [100, 200, 300],
    'classifier__max_depth': [3, 5, 7],
    'classifier__learning_rate': [0.01, 0.1, 0.2],
    'classifier__subsample': [0.8, 0.9, 1.0],
    'classifier__colsample_bytree': [0.8, 0.9, 1.0]
}

# Búsqueda de mejores hiperparámetros
grid_search = GridSearchCV(
    xgb_pipeline, param_grid, cv=5, scoring='roc_auc', n_jobs=-1, verbose=2
)
grid_search.fit(X_train, y_train)

# Mejores parámetros encontrados
print("Mejores parámetros:", grid_search.best_params_)

# Evaluación del modelo con mejores parámetros
best_model = grid_search.best_estimator_
y_pred = best_model.predict(X_test)
y_pred_proba = best_model.predict_proba(X_test)[:, 1]

# Métricas de evaluación
print("\nInforme de clasificación:")
print(classification_report(y_test, y_pred))

print("\nMatriz de confusión:")
conf_matrix = confusion_matrix(y_test, y_pred)
print(conf_matrix)

print("\nÁrea bajo la curva ROC:")
roc_auc = roc_auc_score(y_test, y_pred_proba)
print(roc_auc)

# Guardar predicciones en SQL Server
def guardar_predicciones(modelo, datos, probabilidades):
    conn = conectar_sql_server()
    cursor = conn.cursor()
    
    for i, (id_cliente, prob) in enumerate(zip(datos['ID_Cliente'], probabilidades)):
        nivel_alerta = 'Verde'
        if prob > 0.75:
            nivel_alerta = 'Rojo'
        elif prob > 0.50:
            nivel_alerta = 'Naranja'
        elif prob > 0.20:
            nivel_alerta = 'Amarillo'
        
        query = """
        INSERT INTO ML_Predicciones 
        (ID_Cliente, Fecha_Prediccion, Probabilidad_Morosidad, Nivel_Alerta, Score_Riesgo)
        VALUES (?, GETDATE(), ?, ?, ?)
        """
        
        cursor.execute(query, (int(id_cliente), float(prob), nivel_alerta, int(prob * 100)))
    
    conn.commit()
    cursor.close()
    conn.close()
    print("Predicciones guardadas en SQL Server")

# Guardar modelo entrenado
import joblib
joblib.dump(best_model, 'modelo_xgboost_final.pkl')
print("\nModelo guardado como 'modelo_xgboost_final.pkl'")

# Guardar predicciones en base de datos
guardar_predicciones(best_model, df, y_pred_proba)
\end{verbatim}

\section{Anexo B: Estructura de la Base de Datos}

\begin{table}[ht]
\centering
\begin{tabular}{|p{4cm}|p{3cm}|p{7cm}|}
\hline
\textbf{Tabla} & \textbf{Descripción} & \textbf{Campos principales} \\
\hline
Clientes & Información básica de clientes & ID\_Cliente, Nombre, Región, Categoría, Fecha\_Ingreso \\
\hline
Transacciones & Registro de ventas & ID\_Transacción, ID\_Cliente, Fecha, Monto, Plazo\_Pago \\
\hline
Pagos & Registro de pagos recibidos & ID\_Pago, ID\_Transacción, Fecha\_Pago, Monto \\
\hline
Variables\_Derivadas & Variables calculadas para el modelo & ID\_Cliente, Antigüedad, Máximo\_Días\_Atraso, Índice\_Estacionalidad \\
\hline
Predicciones & Resultados del modelo predictivo & ID\_Cliente, Fecha\_Predicción, Probabilidad\_Morosidad, Nivel\_Alerta \\
\hline
\end{tabular}
\caption{Estructura de la base de datos del sistema}
\end{table}

\section{Anexo C: Configuración del Dashboard}

\begin{figure}[ht]
\begin{verbatim}
# Código de configuración para Power BI (archivo PBIT)

# Conexión a la fuente de datos
Source = Sql.Database(
    "servidor-sqlserver", 
    "db_riesgo_crediticio",
    [
        User = "usuario_analisis",
        HierarchicalNavigation = true
    ]
),

# Consulta principal para mapa de riesgo
QueryMapaRiesgo = 
    "SELECT 
        c.Region, 
        COUNT(p.ID_Cliente) as Num_Clientes,
        AVG(p.Probabilidad_Morosidad) as Riesgo_Promedio,
        SUM(CASE WHEN p.Nivel_Alerta = 'Rojo' THEN 1 ELSE 0 END) as Alertas_Rojas
     FROM Clientes c
     JOIN Predicciones p ON c.ID_Cliente = p.ID_Cliente
     WHERE p.Fecha_Prediccion = 
        (SELECT MAX(Fecha_Prediccion) FROM Predicciones)
     GROUP BY c.Region
     ORDER BY Riesgo_Promedio DESC",
     
# Configuración de actualización
Actualización = 
    Table.AddColumn(
        Source, 
        "Última Actualización", 
        each DateTime.LocalNow()
    )
\end{verbatim}
\caption{Fragmento de configuración del dashboard en Power BI}
\end{figure}

\section{Anexo D: Resultados Detallados por Segmento}

\begin{table}[ht]
\centering
\begin{tabular}{|p{2.5cm}|p{2cm}|p{2cm}|p{2cm}|p{2cm}|p{2cm}|}
\hline
\textbf{Segmento} & \textbf{Precisión} & \textbf{Recall} & \textbf{F1-Score} & \textbf{AUC} & \textbf{Reducción morosidad} \\
\hline
Premium & 0.92 & 0.89 & 0.90 & 0.95 & 22\% \\
\hline
Estable & 0.85 & 0.82 & 0.83 & 0.90 & 19\% \\
\hline
Estacional & 0.76 & 0.79 & 0.77 & 0.84 & 16\% \\
\hline
Alto riesgo & 0.88 & 0.91 & 0.89 & 0.94 & 14\% \\
\hline
\textbf{Global} & \textbf{0.84} & \textbf{0.82} & \textbf{0.83} & \textbf{0.91} & \textbf{18\%} \\
\hline
\end{tabular}
\caption{Resultados detallados por segmento de cliente}
\end{table}

\section{Anexo E: Encuesta de Satisfacción de Usuarios}

\begin{table}[ht]
\centering
\begin{tabular}{|p{5cm}|p{2cm}|p{2cm}|p{2cm}|}
\hline
\textbf{Aspecto evaluado} & \textbf{Gerencia \newline (n=3)} & \textbf{Crédito \newline (n=5)} & \textbf{Cobranzas \newline (n=4)} \\
\hline
Facilidad de uso & 4.3/5 & 4.1/5 & 4.5/5 \\
\hline
Utilidad de predicciones & 4.7/5 & 4.6/5 & 4.8/5 \\
\hline
Claridad de visualizaciones & 4.0/5 & 4.2/5 & 4.3/5 \\
\hline
Impacto en toma de decisiones & 4.3/5 & 4.5/5 & 4.7/5 \\
\hline
Confianza en resultados & 3.7/5 & 4.0/5 & 4.1/5 \\
\hline
\textbf{Satisfacción general} & \textbf{4.2/5} & \textbf{4.3/5} & \textbf{4.5/5} \\
\hline
\end{tabular}
\caption{Resultados de encuesta de satisfacción por grupo de usuarios}
\end{table}

% ======================================
% PÁGINA DE FIRMAS
% ======================================
\newpage
\thispagestyle{empty} % Sin numeración de página
\vspace*{3cm}

\begin{center}
\Large\textbf{FIRMAS DE APROBACIÓN}
\end{center}

\vspace{4cm}

% Firma del Maestrante
\begin{minipage}[t]{0.45\textwidth}
\centering
\rule{8cm}{0.5pt}\\[0.5cm]
\textbf{Firma del Maestrante}\\[0.3cm]
Nombre: \rule{6cm}{0.5pt}\\[0.3cm]
C.I.: \rule{4cm}{0.5pt}\\[0.3cm]
Fecha: \rule{4cm}{0.5pt}
\end{minipage}
\hfill
% Firma del Tutor
\begin{minipage}[t]{0.45\textwidth}
\centering
\rule{8cm}{0.5pt}\\[0.5cm]
\textbf{Firma del Tutor}\\[0.3cm]
Nombre: \rule{6cm}{0.5pt}\\[0.3cm]
C.I.: \rule{4cm}{0.5pt}\\[0.3cm]
Fecha: \rule{4cm}{0.5pt}
\end{minipage}

\vspace{3cm}

% Opción adicional: Firma del Coordinador/Director del Programa (si es necesario)
\begin{center}
\rule{8cm}{0.5pt}\\[0.5cm]
\textbf{Firma del Director del Programa}\\[0.3cm]
Nombre: \rule{6cm}{0.5pt}\\[0.3cm]
C.I.: \rule{4cm}{0.5pt}\\[0.3cm]
Fecha: \rule{4cm}{0.5pt}
\end{center}

\end{document}